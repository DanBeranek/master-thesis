\subsection{Předpoklady}

V této sekci jsou uvedeny základní předpoklady, které byly přijaty při vývoji knihovny \texttt{framesss},
\begin{itemize}
    \item konstrukce je umístěna v pravotočivé souřadné soustavě,
    \item pro kladný směr pootočení platí pravidlo pravé ruky\footnote{Palec natočíme ve směru kladné části osy a zahnuté prsty pravé ruky značí kladný směr otáčení.},
    \item výpočet probíhá podle teorie malých deformací,
    \item materiál je uvažován jako pružný, platí Hookův zákon.
\end{itemize}

Tyto předpoklady umožňují uplatnění principu superpozice.

\subsubsection*{Předpoklady pro model v rovině XZ}

\begin{itemize}
    \item zatížení působí pouze v rovině \gls{X}\gls{Z}, která je zároveň rovinou symetrie,
    \item každý uzel má přiřazené tři stupně volnosti, posun ve směru globální osy \gls{X}, pootočení okolo globální osy \gls{Y} a posun ve směru globální osy \gls{Z}. Kladný směr je naznačený na \autoref{fig:global_dofs},    
    \begin{figure}[H]
        \begin{tikzpicture}[>={Stealth[inset=0pt,length=8pt,angle'=28,round]}]
    \draw[->] (0, 0) -- (2,0) node[above] {\gls{X}, +\gls{u_i}};
    \draw[->] (0, 0) -- (0,-2) node[right] {\gls{Z}, +\gls{w_i}};
    \draw[->,domain=180:450] plot ({cos(\x)}, {sin(\x)}) node[above right] {+\gls{phi_i}};
\end{tikzpicture}
        \caption{Stupně volnosti v globálním systému}
        \label{fig:global_dofs}
    \end{figure}

    \item stupně volnosti a jejich kódová čísla\footnote{Indexování v jazyce Python začíná od $0$, což znamená, že první stupeň volnosti má přiřazené kódové číslo $0$, druhý prvek má kódové číslo $1$ atd.} v lokálním systému prutového prvku jsou označena na \autoref{fig:dof_numbering},
    
    \begin{figure}[H]
        \begin{tikzpicture}[>={Stealth[inset=0pt,length=8pt,angle'=28,round]}]
    \draw[->] (-0.5, -0.5) -- (1,-0.5) node[above] {\gls{X}};
    \draw[->] (-0.5, -0.5) -- (-0.5,1) node[left] {\gls{Z}};

    \draw[dotted,->] (2, 2) -- (9,3.75) node[above] {\gls{x}};
    \draw[dotted,->] (2, 2) -- (1.5,4) node[above] {\gls{z}};
    \point{a}{2}{2};
    \point{b}{6}{3};

    \beam{1}{a}{b}[1][1]

    \draw[fill=white] (2,2) circle (.25cm) node[text=black] {$i$};
    \draw[fill=white] (6,3) circle (.25cm) node[text=black] {$j$};

    \draw[<-] ($(2,2)!0.35cm!(1,1.75)$) -- (0.5, 1.625) node[text=ctublue,above] {0};
    \draw[<-] ($(2,2)!0.35cm!(2.5,0)$) -- (2.375, 0.5) node[text=ctublue,right] {2};
    \draw[->,domain=330:60] plot ({2 + cos(\x)}, {2 + sin(\x)}) node[text=ctublue, above right] {1};

    \draw[->] ($(6,3)!0.35cm!(10,4)$) -- (7.5, 3.375) node[text=ctublue,above] {3};
    \draw[<-] ($(6,3)!0.35cm!(6.5,1)$) -- (6.375, 1.5) node[text=ctublue,right] {5};
    \draw[->,domain=330:60] plot ({6 + cos(\x)}, {3 + sin(\x)}) node[text=ctublue, above right] {4};

\end{tikzpicture}
        \caption{Stupně volnosti v lokálním systému prvku}
        \label{fig:dof_numbering}
    \end{figure}

    \item vnitřní síly v jakémkoliv průřezu prvku jsou: normálová síla \gls{axial_force}, posouvající síla \gls{shear_force} a ohybový moment \gls{bending_moment}. Kladná orientace vnitřních sil je naznačená na \autoref{fig:positive_internal_forces}.
    \begin{figure}[H]
        \begin{tikzpicture}
    \point{a}{0}{0}
    \point{b}{1.5}{0}

    \beam{1}{a}{b}[1][1]

    \load{1}{b}[90][1][-1.1]
    \point{n1}{2.6}{0}
    \notation{1}{n1}{\gls{axial_force}}[above left]

    \load{1}{b}[180][1][-1.1]
    \point{v1}{1.5}{-1.1}
    \notation{1}{v1}{\gls{shear_force}}[above right]

    \load{3}{b}[225]
    \point{m1}{1.4}{0.3}
    \notation{1}{m1}{\gls{bending_moment}}[above left]

    \point{c}{4.5}{0}
    \point{d}{6}{0}
    \beam{1}{c}{d}[1][1]

    \load{1}{c}[0][1][-1.1]
    \point{n2}{3.4}{0}
    \notation{1}{n2}{\gls{axial_force}}[above right]

    \load{1}{c}[270][1][-1.1]
    \point{v2}{4.5}{1}
    \notation{1}{v2}{\gls{shear_force}}[below right]

    \load{2}{c}[135]
    \point{m2}{4.5}{-0.4}
    \notation{1}{m2}{\gls{bending_moment}}[below]

\end{tikzpicture}
        \caption{Kladná orientace vnitřních sil}
        \label{fig:positive_internal_forces}
    \end{figure}
\end{itemize}

