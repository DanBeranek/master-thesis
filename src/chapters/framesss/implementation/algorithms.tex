\subsection{Algoritmizace}

V této sekci je představen algoritmus, který je jádrem metody konečných prvků implementované v knihovně \texttt{framesss}. Tento algoritmus systematicky zpracovává model od jeho diskretizace až po výpočet odezvy konstrukce na zatížení jednotlivými zatěžovacími stavy a jejich kombinacemi.

\begin{algorithm}[H]
    \caption{Proces výpočtu modelu}
    \DontPrintSemicolon  % Don't print semicolons
    
    \SetKwFunction{FMain}{solve()}
    \SetKwProg{Fn}{Metoda}{:}{}
    \Fn{\FMain}{
        Diskretizace prutů na jednotlivé elementy\;
        Očíslování stupňů volnosti\;
        Lokalizace globální matice tuhosti\;
        \For{každý zatěžovací stav}{
            Inicializace vektoru koncových posunů\;
            Lokalizace předepsaných deformací\;
            Inicializace vektoru zatížení\;
            Lokalizace sil v uzlech\;
            Lokalizace koncových sil na prvcích\; 
            Výpočet vektoru neznámých koncových posunů\;
            Výpočet reakcí\;
            Výpočet koeficientů vnitřních sil\;
            Výpočet extrémů vnitřních sil\;
            Uložení vypočítaných hodnot\;
        }
        \For{každá kombinace zatížení}{
            Inicializace výsledků\;
            \For{každý zatěžovací stav v kombinaci}{
                Přičtení hodnot k výsledkům\;
            }
            Uložení výsledků\;
        }
        \For{každá obálka}{
            Inicializace\;
            Určení extrémních hodnot v každé kombinaci\;
            Uložení hodnot\;
        }
        \Return{}
    }
    
\end{algorithm}

\subsection{Řídké matice}

V rámci knihovny \texttt{framesss} je zásadní efektivní manipulace s řídkými maticemi, které se často vyskytují v důsledku velkého množství nulových prvků v maticích tuhosti. Pro ukládání a zpracování těchto matic využívá \texttt{framesss} knihovnu SciPy \cite{scipy}, specificky její modul pro řídké matice.

\subsubsection{Formáty ukládání řídkých matic}

V rámci knihovny SciPy jsou k dispozici různé formáty pro ukládání řídkých matic, z nichž každý je optimalizován pro specifické typy operací.

\begin{itemize}
    \item \textbf{CSR (Compressed Sparse Row)}: Ukládá data kompresí řádkových indexů. Tento formát je efektivní pro řádkové operace.
    \item \textbf{CSC (Compressed Sparse Column)}: Podobné jako CSR, ale komprimuje sloupcové indexy. Je efektivní pro sloupcové operace.
    \item \textbf{BSR (Block Sparse Row)}: Velmi podobný formátu CSR. Je vhodný pro řídké matice s hustými submaticemi.
    \item \textbf{COO (COOrdinate format)}: Ukládá seznam trojic (řádkový index, sloupcový index a hodnotu), což umožňuje efektivní postupné sestavování řídkých matic.
    \item \textbf{DIA (DIAgonal storage)}: Ukládá hodnoty na diagonálách v samostatných polích a jejich offset od diagonály, což je užitečné pro matice s dominantními diagonálami.
    \item \textbf{DOK (Dictionary Of Keys)}: Ukládá prvky matice ve slovníku, kde klíče jsou dvojice (řádek, sloupec).
    \item \textbf{LIL (LIst of Lists)}: Ukládá matice pomocí dvou seznamů, jeden pro řádky, ve kterých jsou ukládány sloupcové indexy, a druhý pro hodnoty.
\end{itemize}


\subsubsection{Využití ve \texttt{framesss}}

Proces vytváření a manipulace s řídkými maticemi v knihovně \texttt{framesss} zahrnuje několik klíčových kroků, které využívají různé formáty matic pro optimalizaci různých fází výpočtu:

\begin{enumerate}
    \item \textbf{Vytvoření matice jako COO}: Ve fázi lokalizace je globální matice tuhosti vytvářena jako COO, což je formát efektivní pro konstrukci řídkých matic
    
    \item \textbf{Převod do LIL}: Po sestavení základní struktury matice je převedena do formátu LIL pro rychlé řezání (slicing), indexování a manipulaci. LIL umožňuje efektivní změny řádků a sloupců matice, což je vyžadováno během dělení soustavy rovnic na dvě soustavy rovnic.
    
    \item \textbf{Převod do CSR}: Konečný převod matice do formátu CSR je proveden pro rychlé aritmetické operace, které jsou nezbytné pro  řešení systému rovnic. CSR formát je vysoce optimalizovaný pro rychlé maticové operace.
\end{enumerate}

Každý z těchto formátů má specifické výhody pro různé typy operací a jejich výběr je motivován potřebou maximalizovat výpočetní efektivitu a minimalizovat dobu provádění v různých fázích výpočetního procesu. Použití COO pro počáteční sestavení, LIL pro manipulaci s řádky a sloupci a CSR pro finální výpočty umožňuje dosáhnout optimálního výkonu.

