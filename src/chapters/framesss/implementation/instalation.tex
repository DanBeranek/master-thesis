\subsection{Instalace} \label{sec:framesss_instalation}

Knihovna \texttt{framesss} je navržena tak, aby byla snadno dostupná a uživatelsky přívětivá nejen z hlediska jejího používání, ale i instalace. Pro instalaci této knihovny můžete využít \texttt{pip}, což je standardní správce balíčků pro Python, který zjednodušuje správu softwarových závislostí a zaručuje rychlé a snadné nasazení knihovny \texttt{framesss}.

Pro instalaci knihovny \texttt{framesss} stačí spustit následující příkaz v terminálu:

\begin{verbatim}
pip install framesss
\end{verbatim}

Tento příkaz stáhne nejnovější verzi knihovny \texttt{framesss} z Python Package Index (PyPI) a nainstaluje ji spolu s nezbytnými závislostmi. Před instalací se ujistěte, že máte na svém systému již nainstalovaný Python a \texttt{pip}. Pokud narazíte během instalace na jakékoli problémy, zkuste nejprve aktualizovat \texttt{pip} pomocí příkazu \texttt{pip install --upgrade pip} a poté pokračujte v reinstalaci balíčku.

Pro uživatele, kteří potřebují integrovat knihovnu \texttt{framesss} do větších projektů nebo virtuálních prostředí, doporučujeme instalaci v rámci Python virtuálního prostředí. Tento postup pomáhá lépe spravovat závislosti a udržuje vaše Python projekty organizované a bez konfliktů.

Pro instalaci knihovny \texttt{framesss} v Python virtuálním prostředí, postupujte podle následujících kroků pro váš operační systém.


\begin{itemize}
    \item \textbf{Pro Unixové systémy (Linux/Mac):}
        \begin{verbatim}
        python -m venv framesss-env
        source framesss-env/bin/activate
        pip install framesss
        \end{verbatim}
    
    \item \textbf{Pro Windows:}
        \begin{verbatim}
        python -m venv framesss-env
        framesss-env\Scripts\activate
        pip install framesss
        \end{verbatim}
\end{itemize}

Po instalaci můžete knihovnu \texttt{framesss} importovat do svých Python skriptů následovně:

\begin{verbatim}
import framesss
\end{verbatim}

