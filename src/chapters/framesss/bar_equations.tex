\section{Prutové prvky}

V této části se zaměříme na řešení prutových konstrukcí. Prut je těleso s výrazně převládající délkou nad rozměry průřezu. Dimenzi z pohledu napjatosti budeme redukovat na jednorozměrný problém. 

Přijmeme následující předpoklady,
\begin{itemize}
    \item zatížení působí pouze v rovině \gls{X}\gls{Z}, která je i rovinou symetrie, řešení není funkcí souřadnice \gls{y},
    \item průhyb ve směru osy \gls{z} je po výšce prvku konstantní,
    \item výrazně převládajícím rozměrem je délka prvku \gls{L},
    \item uvažujeme prizmatický prut (s neměnným průřezem po délce prvku),
    \item uvažujeme teorii malých přetvoření,
    \item materiál prutu se chová pružně a je homogenní.
\end{itemize}

\subsubsection*{Vektor posunutí}
Vzhledem k přijatým předpokladům je posun ve směru osy \gls{Y} nulový, zároveň uvažujeme po výšce nestlačitelný prut, to znamená, že posun ve směru osy \gls{z} je závislý pouze na souřadnici \gls{x}. V dalších kapitolách si vystačíme s vektorem posunutí ve tvaru

\begin{equation}
    \gls{u}(\gls{x_}) = \begin{Bmatrix}
        \gls{u_i}[ ](\gls{x}, \gls{z}) &
        \gls{w_i}[ ](\gls{x})
    \end{Bmatrix}{^\mathrm{T}}.
\end{equation}

\subsubsection*{Vektor deformací}
Vektor deformací odpovídající předpokladům má dvě nenulové složky, 

\begin{equation}
    \gls{eps}(\gls{x}, \gls{z})
    = 
    \begin{Bmatrix}
        \gls{eps_i}[\gls{x}](\gls{x}, \gls{z}) &
        \gls{gamma_i}[\gls{z}\gls{x}](\gls{x}, \gls{z})
    \end{Bmatrix}^{\mathrm{T}}
    =
    \begin{Bmatrix}
        \pdv{\gls{u_i}[ ](\gls{x}, \gls{z})}{\gls{x}} &
        \pdv{\gls{u_i}[ ](\gls{x}, \gls{z})}{\gls{z}} + \pdv{\gls{w_i}[ ](\gls{x})}{\gls{x}}
    \end{Bmatrix}^{\mathrm{T}}.
\end{equation}


\subsubsection*{Vektor napětí}

Při analýze prutových konstrukcí je běžné pracovat s vnitřními silami místo s napětím. Vnitřní síly jsou s napětím svázany podmínkami ekvivalence
\begin{align}
    \gls{axial_force}(\gls{x}) &= \int_{\gls{A}} \gls{sigma_i}[\gls{x}] \dd{\gls{A}}, \label{eq:N}\\
    \gls{shear_force}(\gls{x}) &= \int_{\gls{A}} \gls{tau_i}[\gls{x}\gls{z}] \dd{\gls{A}}, \label{eq:V}\\
    \gls{bending_moment}(\gls{x}) &= \int_{\gls{A}} \gls{sigma_i}[\gls{x}] \gls{z} \dd{\gls{A}}, \label{eq:M}
\end{align}
kde $\dd{\gls{A}} = \dd{\gls{y}}\dd{\gls{z}}$.

\subsubsection*{Podmínky rovnováhy}
Podmínky rovnováhy na prvku lze zapsat jako
\begin{align}
    \dv{\gls{axial_force}(\gls{x})}{\gls{x}} + \bar{f}_{\gls{x}}(\gls{x}) &= 0, \\
    \dv{\gls{shear_force}(\gls{x})}{\gls{x}} + \bar{f}_{\gls{z}}(\gls{x}) &= 0, \label{eq:V_equilibrium}\\
    \dv{\gls{bending_moment}(\gls{x})}{\gls{x}} - \gls{shear_force}(\gls{x}) &= 0, \label{eq:M_equilibrium}
\end{align}
kde $\bar{f}_{\gls{x}}(\gls{x})$ a $\bar{f}_{\gls{z}}(\gls{x})$ je zatížení působící po délce prutu.

\subsubsection*{Kinematické podmínky}
Kinematické podmínky jsou dány typem podpory, např. pro vetknutí platí
\begin{equation}
    \gls{u_i}[ ] = 0, \quad \gls{w_i}[ ] = 0, \quad \gls{phi_i}[ ] = 0.
\end{equation}

\subsection{Tažený-tlačený prut}

\begin{figure}[H]
    \newcommand*{\xStart}{0}
\newcommand*{\xEnd}{8}
\newcommand*{\uStart}{1}
\newcommand*{\uEnd}{2}
\newcommand*{\height}{2}
\newcommand*{\xDivisions}{10} % Number of divisions along the x-axis
\newcommand*{\yDivisions}{4}  % Number of divisions along the y-axis

\begin{tikzpicture}[>={Stealth[inset=0pt,length=8pt,angle'=28,round]}, scale=0.8]
    % Draw axes
    \draw[->] (0, 0) -- (11, 0) node[above] {\gls{x}};
    \draw[->] (0, 0) -- (0, 2.5) node[above] {\gls{z}};
    
    % Draw the initial bar
    \draw[black] (\xStart, -\height/2) rectangle (\xEnd, \height/2);

    % Calculate spacing
    \pgfmathsetmacro{\xSpacing}{(\xEnd-\xStart)/\xDivisions}
    \pgfmathsetmacro{\ySpacing}{\height/\yDivisions}
    \pgfmathsetmacro{\xDefDivs}{\xDivisions-1}
    \pgfmathsetmacro{\yDefDivs}{\yDivisions-1}

    %Draw bar grid
    \foreach \i in {1,...,\xDefDivs} {
        \draw[draw=black, dotted] (\xStart + \i*\xSpacing, -\height/2) -- (\xStart + \i*\xSpacing, \height/2); % Vertical grid lines
    }
    \foreach \i in {1,...,\yDefDivs} {
        \draw[draw=black, dotted] (\xStart, -\height/2 + \i*\ySpacing) -- (\xEnd, -\height/2 + \i*\ySpacing); % Horizontal grid lines
    }

    % Draw deformed bar
    \draw[ctublue, thick] (\xStart + \uStart, -\height/2) rectangle (\xEnd + \uEnd, \height/2);
    
    % Draw deformed grid
    \pgfmathsetmacro{\uDelta}{(\uEnd - \uStart)/(\xDivisions)}

    \foreach \i in {1,...,\xDefDivs} {
        \draw[ctulightblue] (\xStart + \i*\xSpacing + \uStart + \i*\uDelta, -\height/2) -- (\xStart + \i*\xSpacing + \uStart + \i*\uDelta, \height/2); % Vertical deformed grid lines
    }
    \foreach \i in {1,...,\yDefDivs} {
        \draw[ctulightblue] (\xStart + \uStart, -\height/2 + \i*\ySpacing) -- (\xEnd + \uEnd, -\height/2 + \i*\ySpacing); % Horizontal deformed grid lines
    }
    % nodes
    \draw[ultra thick] (\xStart, 0) -- (\xEnd, 0);
    \draw[fill=white] (\xStart, 0) circle (.25cm) node[text=black] {$a$};
    \draw[fill=white] (\xEnd, 0) circle (.25cm) node[text=black] {$b$};

    \draw[|->|, draw=ctublue, text=ctublue] (\xStart, +\height/2+0.3) -- node[above] {$u_a$} ++(\uStart, 0);
    \draw[|->|, draw=ctublue, text=ctublue] (\xEnd, +\height/2+0.3) -- node[above] {$u_b$} ++(\uEnd, 0);

    \draw[|<->|, draw=black] (\xStart, -\height/2-0.7) -- node[above] {\gls{L}} (\xEnd, -\height/2-0.7);
    

\end{tikzpicture}
    \caption{Deformovaná konfigurace taženého-tlačeného prutu}
    \label{fig:deformed_bar}
\end{figure}

Tažený-tlačený prut je v rovině dán pomocí dvou uzlů, $a$ a $b$. V každém bodě zavedeme koncové posunutí ve směru lokální osy \gls{x}. Vektor zobecněných posunutí má tvar
\begin{equation}
    \gls{d} = \begin{Bmatrix}
        \gls{u_i}[a] \\
        \gls{u_i}[b]
    \end{Bmatrix}.
\end{equation}

Pomocí vektoru koncových posunutí \gls{d} a matice bázových funkcí \gls{N} budeme aproximovat posunutí $\gls{u_i}[ ](\gls{x})$ libovolného průřezu prvku,
\begin{equation} \label{eq:bar_ux}
    \gls{u_i}[ ](\gls{x}) \approx \gls{N}(\gls{x}) \gls{d}
    = \begin{bmatrix}
        \gls{N_i}[1](\gls{x}) & \gls{N_i}[2](\gls{x})
    \end{bmatrix}
    \begin{Bmatrix}
        \gls{u_i}[a] \\
        \gls{u_i}[b]
    \end{Bmatrix}.
\end{equation}

\subsubsection*{Bázové funkce}

Bázové funkce $\gls{N_i}[i](\gls{x})$ budeme hledat ve tvaru polynomu prvního stupně,
\begin{equation}
    \gls{N_i}[i](\gls{x}) = a_i \gls{x} + b_i,
\end{equation}
kde $a_i$ a $b_i$ jsou zatím neznámé koeficienty. Rozepsáním vztahu \ref{eq:bar_ux} do maticové podoby dostaneme výraz
\begin{equation} \label{eq:bar_ux_matrix}
    \gls{u_i}[ ](\gls{x})
    =
    \begin{bmatrix}
        \gls{x} & 1
    \end{bmatrix}
    \begin{bmatrix}
        a_1 & a_2 \\
        b_1 & b_2
    \end{bmatrix}
    \begin{Bmatrix}
        \gls{u_i}[a] \\
        \gls{u_i}[b]
    \end{Bmatrix}.
\end{equation}


Okrajové podmínky, patrné z obrázku \ref{fig:deformed_bar} jsou
\begin{subequations}
    \begin{equation} \label{eq:bar_ua}
        \gls{u_i}[ ](0) = \gls{u_i}[a], \\        
    \end{equation}
    \begin{equation} \label{eq:bar_ub}
        \gls{u_i}[ ](\gls{L}) = \gls{u_i}[b].
    \end{equation}
\end{subequations}

Dosazením okrajových podmínek \ref{eq:bar_ua} a \ref{eq:bar_ub} do vztahu \ref{eq:bar_ux_matrix} dostaneme soustavu rovnic
\begin{equation}
    \underbrace{
    \begin{bmatrix}
        0 & 1 \\
        \gls{L} & 1
    \end{bmatrix}}_\matr{A}
    \begin{bmatrix}
        a_1 & a_2 \\
        b_1 & b_2
    \end{bmatrix}
    \begin{Bmatrix}
        \gls{u_i}[a] \\
        \gls{u_i}[b]
    \end{Bmatrix} = 
    \begin{Bmatrix}
        \gls{u_i}[a] \\
        \gls{u_i}[b]
    \end{Bmatrix}.
\end{equation}

Přenásobením rovnice zleva maticí $\matr{A}^{-1}$ dostaneme
\begin{equation}
    \begin{bmatrix}
        a_1 & a_2 \\
        b_1 & b_2
    \end{bmatrix}
    \begin{Bmatrix}
        \gls{u_i}[a] \\
        \gls{u_i}[b]
    \end{Bmatrix} 
    = 
    \begin{bmatrix}
        0 & 1 \\
        \gls{L} & 1
    \end{bmatrix}^{-1}
    \begin{Bmatrix}
        \gls{u_i}[a] \\
        \gls{u_i}[b]
    \end{Bmatrix},
\end{equation}
ze které je patrné, že matice koeficientů bázových funkcí se rovná inverzní matici k matici $\matr{A}$,
\begin{equation}
    \begin{bmatrix}
        a_1 & a_2 \\
        b_1 & b_2
    \end{bmatrix}
    =
    \begin{bmatrix}
        -\frac{1}{\gls{L}} & \frac{1}{\gls{L}} \\
        1 & 0
    \end{bmatrix}.
\end{equation}

Bázové funkce, aproximující posunutí \gls{u_i}[ ](\gls{x}) po délce prvku, tedy jsou
\begin{subequations}
    \begin{equation}
        \gls{N_i}[1](\gls{x}) = 1 -\frac{\gls{x}}{\gls{L}},
    \end{equation}
    \begin{equation}
        \gls{N_i}[2](\gls{x}) = \frac{\gls{x}}{\gls{L}}.
    \end{equation}
\end{subequations}

\begin{figure}[H]
    \begin{tikzpicture}[>={Stealth[inset=0pt,length=8pt,angle'=28,round]},]
    % N1
    \draw[draw=black] (0.0, 0.0) node[below] {0} -- (5.0, 0.0) node[below] {\gls{L}};
    \draw[color=ctublue, thick] (0, 1) -- (5, 0);
    \draw[->] (0, 0) -- (0, 1);
    \node[rotate=90, text=ctublue, above] at (0, 0.5) {$1$};
    \node[text=ctublue, above] at (2.5, 0.5) {$N_1(x)$};

    % N2
    \draw[draw=black] (6, 0) node[below] {0} -- (11, 0) node[below] {\gls{L}};
    \draw[color=ctublue, thick] (6, 0) -- (11, 1);
    \draw[->] (11, 0) -- (11, 1);
    \node[rotate=90, text=ctublue, above] at (11, 0.5) {$1$};
    \node[text=ctublue, above] at (8.5, 0.5) {$N_2(x)$};
\end{tikzpicture}
    \caption{Bázové funkce taženého-tlačeného prutu}
    \label{fig:bar_shape_functions}
\end{figure}

Dále vypočítáme matici \gls{B}, která popisuje vztah mezi posuny a poměrným přetvořením. Pro přetvoření taženého-tlačeného prvku platí vztah
\begin{equation}
    \gls{eps_i}[\gls{x}](\gls{x}) = \dv{\gls{u_i}[ ](\gls{x})}{\gls{x}} \approx \dv{\gls{N}(\gls{x}) \gls{d}}{\gls{x}} = \gls{B}(\gls{x})\gls{d}.
\end{equation}
V matici \gls{B} se tedy vyskytují první derivace bázových funkcí \gls{N_i}[i] podle \gls{x},
\begin{equation} \label{eq:bar_B}
    \gls{B}(\gls{x}) = \dv{\gls{N}(\gls{x})}{\gls{x}}
    =
    \dv{\gls{x}} 
    \Bigg[
        \begin{array}{cc}
            1 - \dfrac{\gls{x}}{\gls{L}} & \dfrac{\gls{x}}{\gls{L}}
        \end{array}
    \Bigg]
    =
    \Bigg[
        \begin{array}{cc}
            -\dfrac{1}{\gls{L}} & \dfrac{1}{\gls{L}}
        \end{array}
    \Bigg]
    =
    \frac{1}{\gls{L}}
    \begin{bmatrix}
        -1 & 1
    \end{bmatrix}.
\end{equation}

\subsubsection*{Matice tuhosti}
Pro výpočet matice tuhosti využijeme vztah \ref{eq:K}
\begin{align*}
    \gls{K} &= \int_{\gls{omega}} \gls{B}^{\mathrm{T}}\gls{D}\gls{B} \dd{\gls{omega}},
\end{align*}
integrál přes oblast \gls{omega} rozdělíme na integrál přes průřez a integrál po délce prutu,
\begin{align*}
    \gls{K}  &= \int_{0}^{\gls{L}} \int_{\gls{A}} \gls{B}^{\mathrm{T}}\gls{D}\gls{B} \dd{\gls{A}} \dd{\gls{x}},
\end{align*}
výraz $\gls{B}^{\mathrm{T}}\gls{D}\gls{B}$ nezávisí na průřezové ploše, můžeme jej tedy vytknout před vnitřní integrál,
\begin{align*}
    \gls{K}  &= \int_{0}^{\gls{L}} \gls{B}^{\mathrm{T}}\gls{D}\gls{B} \int_{\gls{A}} \dd{\gls{A}} \dd{\gls{x}},
\end{align*}
uvažujeme prizmatický prut (s konstantním průřezem), což znamená, že integrál $\int_{\gls{A}}\dd{\gls{A}} = \gls{A}$,
\begin{align*}
    \gls{K} &=\int_{0}^{\gls{L}} \gls{B}^{\mathrm{T}}\gls{D}\gls{B} \gls{A} \dd{\gls{x}},
\end{align*}
matice materiálové tuhosti \gls{D} při jednoosém namáhání odpovídá modulu pružnosti \gls{E},
\begin{align*}
    \gls{K} &=\int_{0}^{\gls{L}} \gls{B}^{\mathrm{T}}\gls{E}\gls{B} \gls{A} \dd{\gls{x}},
\end{align*}
modul pružnosti \gls{E} a průřezová plocha \gls{A} nejsou funkcí proměnné \gls{x}, můžeme je tedy vytknout před integrál,
\begin{align*}
    \gls{K} &= \gls{E} \gls{A} \int_{0}^{\gls{L}} \gls{B}^{\mathrm{T}}\gls{B} \dd{\gls{x}},
\end{align*}
dosazením za \gls{B} z rovnice \ref{eq:bar_B} dostaneme
\begin{align*}
    \gls{K} &= \gls{E} \gls{A} \int_{0}^{\gls{L}} \frac{1}{\gls{L}^2} 
    \begin{bmatrix} -1 \\ 1 \end{bmatrix}
    \begin{bmatrix} -1 & 1 \end{bmatrix} \dd{\gls{x}},
\end{align*}
délka prvku \gls{L} také není funkcí proměnné \gls{x}, opět ji můžeme vytknout před integrál. Zároveň roznásobíme matice,
\begin{align*}
    \gls{K} &= \frac{\gls{E}\gls{A}}{\gls{L}^2}\int_{0}^{\gls{L}}
    \begin{bmatrix}
    1 & -1 \\
    -1 & 1    
    \end{bmatrix}
    \dd{\gls{x}},
\end{align*}
integrací dostaneme,
\begin{align*}
    \gls{K} &= \frac{\gls{E}\gls{A}}{\gls{L}^2}
    \Bigg[
    \begin{bmatrix}
    \gls{x} & -\gls{x} \\
    -\gls{x} & \gls{x}    
    \end{bmatrix}
    \Bigg]_{0}^{\gls{L}},
\end{align*}
a nakonec dosazením horní a spodní meze obdržíme známý výraz pro matici tuhosti taženého-tlačeného prvku,
\begin{equation}
    \gls{K} = \frac{\gls{E}\gls{A}}{\gls{L}}
    \begin{bmatrix}
        1 & -1 \\
        -1 & 1
    \end{bmatrix}.
\end{equation}

\subsubsection*{Vektor zatížení}

Předpokládejme, že tažený-tlačený prvek je zatížený lineárně se měnícím spojitým zatížením působícím v těžišťové ose,
\begin{equation}
    \overline{f_{\gls{x}}}(\gls{x}) = \frac{\overline{f}_b - \overline{f}_a}{\gls{L}} \gls{x} + \overline{f_a},
\end{equation}
\begin{figure}[H]
    \newcommand*{\xStart}{0}
\newcommand*{\xEnd}{8}
\newcommand*{\height}{2}
\newcommand*{\offset}{0.35}
\newcommand*{\fStart}{1}
\newcommand*{\fEnd}{2.5}
\newcommand*{\xDivisions}{5} % Number of divisions along the x-axis

\begin{tikzpicture}[>={Stealth[inset=0pt,length=8pt,angle'=28,round]},]
    % Draw axes
    \draw[->] (0, 0) -- (9.5, 0) node[above] {\gls{x}};
    \draw[->] (0, 0) -- (0, 2.5) node[above] {\gls{z}};
    
    \draw[|-|, ultra thick] (\xStart, 0) -- (\xEnd, 0);
    \draw[fill=white] (\xStart-\offset, -\offset) circle (.25cm) node[text=black] {$a$};
    \draw[fill=white] (\xEnd+\offset, -\offset) circle (.25cm) node[text=black] {$b$};

    % Calculate spacing
    \pgfmathsetmacro{\xSpacing}{(\xEnd-\xStart)/\xDivisions}
    \pgfmathsetmacro{\xOffset}{0.05*\xSpacing}
    %Draw bar grid
    \foreach \i in {1,...,\xDivisions} {
        \pgfmathsetmacro{\j}{\i-1}
        \draw[->, draw=ctublue, thick] (\xStart + \j*\xSpacing, 0.2) -- (\xStart + \i*\xSpacing - \xOffset, 0.2); % Arrows along the bar
    }
    \draw[draw=ctublue, thick] (\xStart, 0) -- (\xStart, \fStart) -- node[above, text=ctublue] {$\overline{f_{x}}(x)$} (\xEnd, \fEnd) -- (\xEnd, 0);

    \draw[|->|, draw=ctublue] (-0.5, 0) -- node[rotate=90, text=ctublue, above] {$\overline{f}_a$} (-0.5, \fStart);
    \draw[|->|, draw=ctublue] (\xEnd+0.7, 0) -- node[rotate=90, text=ctublue, above] {$\overline{f}_b$} (\xEnd+0.7, \fEnd);
    \draw[|<->|, draw=black] (\xStart, -0.7) -- node[above] {\gls{L}} (\xEnd, -0.7);
\end{tikzpicture}
    \caption{Zatížení taženého-tlačeného prutu}
    \label{fig:bar_load}
\end{figure}

Pro vyjádření vektoru zatížení použijeme dříve odvozený vztah \ref{eq:f},
\begin{equation*}
    \gls{f} = \int_{\gls{omega}} \gls{N}^{\mathrm{T}} \gls{X_bar} \dd{\gls{omega}} + \int_{\gls{gamma_p}} \gls{N}^{\mathrm{T}} \gls{p_bar} \dd{\gls{gamma_p}},
\end{equation*}
kde $\gls{X_bar} = \matr{0}$ a $\gls{p_bar} = \begin{Bmatrix}\bar{f_{\gls{x}}} & 0 & 0\end{Bmatrix}^{\mathrm{T}}$.
Vztah \ref{eq:f} se zjednodušší na
\begin{equation}
    \gls{f} = \int_{0}^{\gls{L}} \gls{N}^{\mathrm{T}} \bar{f_{\gls{x}}} \dd{\gls{x}}.
\end{equation}

Dosazením za \gls{N} a $\overline{f_{\gls{x}}}$ dostaneme,

\begin{equation}
    \gls{f} 
    = \int_{0}^{\gls{L}} 
    \begin{bmatrix}
        1 - \dfrac{\gls{x}}{\gls{L}} \\ \dfrac{\gls{x}}{\gls{L}} 
    \end{bmatrix}
    \left( \frac{\overline{f}_b - \overline{f}_a}{\gls{L}} \gls{x} + \overline{f_a} \right)
    \dd{\gls{x}},
\end{equation}

integrací dostaneme,

\begin{equation}
    \gls{f} 
    =
    \left[
    \begin{bmatrix}
        \overline{f}_a \gls{x} 
        + \dfrac{\gls{x}^2 (-2\overline{f}_a + \overline{f}_b)}{2\gls{L}}
        + \dfrac{\gls{x}^3 (\overline{f}_a - \overline{f}_b)}{3\gls{L}^2}
        \\
        \dfrac{\overline{f}_a \gls{x}^2}{2 \gls{L}}
        + \dfrac{\gls{x}^3(-\overline{f}_a+\overline{f}_b)}{3\gls{L}^2}
    \end{bmatrix}
    \right]_{0}^{\gls{L}}.
\end{equation}

Po dosazení mezí a úpravě se výraz zjednoduší a obdržíme vztah pro vektor zatížení \gls{f},

\begin{equation}
    \gls{f}
    =
    \begin{bmatrix}
        \dfrac{\gls{L}(2\overline{f}_a+\overline{f}_b)}{6} \\
        \dfrac{\gls{L}(\overline{f}_a+2\overline{f}_b)}{6}
    \end{bmatrix}
\end{equation}
