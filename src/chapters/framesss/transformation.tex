\section{Transformační matice}

Matice tuhosti a vektory zatížení jsme v předešlých kapitolách odvodili za předpokladu, že lokální souřadnicový systém je shodně orientovaný s globálním souřadnicovým systémem. Aby bylo možné sestavit podmínky rovnováhy na celé konstrukci, je nutné pracovat v globálním systému souřadnic.

\begin{figure}[H]
    \begin{tikzpicture}[>={Stealth[inset=0pt,length=12pt,angle'=28,round]}, scale=0.8]
    \draw[->, thick] (0, 0) -- ++(12, 0) node[above] {\gls{X}};
    \draw[->, thick] (0, 0) -- ++(0, 6) node[above] {\gls{Z}};

    \draw[->, ctulightblue] (3, 2) -- ++(8, 4) node[above] {\gls{x}};
    \draw[->, ctulightblue] (3, 2) -- ++(-2, 4) node[above] {\gls{z}};
    \draw[very thick, draw=ctublue] (3, 2) node[below left] {$a$} -- ++(6, 3) node[below right] {$b$};

    \draw[dotted] (3, 0) node[below] {$\gls{X}_{a}$} -- ++(0, 2) -- ++(-3, 0) node[left] {$\gls{Z}_{a}$};
    \draw[dotted] (3, 5) -- ++(0, -3) -- ++(6, 0);
    \draw[dotted] (9, 0) node[below] {$\gls{X}_{b}$} -- ++(0, 5) -- ++(-9, 0) node[left] {$\gls{Z}_{b}$};

    \draw[<->,domain=0:26.565, draw=ctulightblue] plot ({3+4*cos(\x)}, {2+4*sin(\x)}) node[below left=0.5 and -0.2] {$\theta_x$};
    \draw[<->,domain=26.565:90, draw=ctulightblue] plot ({3+2.5*cos(\x)}, {2+2.5*sin(\x)}) node[below right=0.35 and 0.5] {$\theta_z$};

    \filldraw[black] (5.5, 3.25) circle (2pt) node[above] {P};
    \draw[dotted] (5.5, 0) node[below] {$\gls{X}_{\mathrm{P}}$} -- ++(0, 3.25) -- ++(-5.5, 0) node[left] {$\gls{Z}_{\mathrm{P}}$};
\end{tikzpicture}
    \caption{Transformace souřadnic}
    \label{fig:coordinate_transformation}
\end{figure}

Polohu bodu $\mathrm{P}$ na obr. \ref{fig:coordinate_transformation} v lokálním souřadném systému lze vyjádřit pomocí vztahu
\begin{equation} \label{eq:P_transformation}
    \gls{x_}_{\mathrm{P},\ell} = \gls{T}  \gls{x_}_{\mathrm{P},g},
\end{equation}
kde \gls{T} je transformační matice, $\gls{x_}_{\mathrm{P},\ell}$ je polohový vektor v lokálním souřadném systému \gls{x}\gls{y}\gls{z} a $\gls{x_}_{\mathrm{P},g}$ je polohový vektor v globálním souřadném systému \gls{X}\gls{Y}\gls{Z}.

Rovnici \ref{eq:P_transformation} lze pro lepší představu rozepsat po složkách,
\begin{equation}
    \begin{Bmatrix}
        \gls{x}_{\mathrm{P}} \\
        \gls{y}_{\mathrm{P}} \\
        \gls{z}_{\mathrm{P}}
    \end{Bmatrix}
    =
    \begin{bmatrix}
        \matr{e}_{\gls{x}} &
        \matr{e}_{\gls{y}} &
        \matr{e}_{\gls{z}}
    \end{bmatrix}^{\mathrm{T}}
    \begin{Bmatrix}
        \gls{X}_{\mathrm{P}} \\
        \gls{Y}_{\mathrm{P}} \\
        \gls{Z}_{\mathrm{P}}
    \end{Bmatrix},
\end{equation}
kde $\matr{e}_i$ označuje jednotkové vektory ve směru lokálních souřadnicových os v globálním systému souřadnic.

Vektor $\matr{e}_{\gls{x}}$ je ve 3D kartézské soustavě souřadnic jednoznačně dán polohou bodů $a$ a $b$,
\begin{equation}
    \matr{e}_{\gls{x}} = 
    \begin{Bmatrix}
        \cos({\theta_{\gls{x}}}) &
        \cos({\theta_{\gls{y}}}) &
        \cos({\theta_{\gls{z}}})
    \end{Bmatrix}^{\mathrm{T}},
\end{equation}
kde
\begin{equation}
    \begin{gathered}
        \cos({\theta_{\gls{x}}}) = \frac{\gls{X}_b - \gls{X}_a}{\gls{L}} = \frac{\Delta \gls{X}}{\gls{L}} \\
        \cos({\theta_{\gls{y}}}) = \frac{\gls{Y}_b - \gls{Y}_a}{\gls{L}} = \frac{\Delta \gls{Y}}{\gls{L}} \\
        \cos({\theta_{\gls{z}}}) = \frac{\gls{Z}_b - \gls{Z}_a}{\gls{L}} = \frac{\Delta \gls{Z}}{\gls{L}} \\
        \gls{L} = \sqrt{\left(\Delta \gls{X}\right)^2 + \left(\Delta \gls{Y}\right)^2 + \left(\Delta \gls{Z}\right)^2}
    \end{gathered}
\end{equation}

Směrové vektory $matr{e}_{\gls{y}}$ a $\matr{e}_{\gls{z}}$ se určí pomocí dalšího bodu, případně vektoru, ležíciho v lokální rovině $\gls{x}\gls{y}$ nebo $\gls{x}\gls{z}$.
Při výpočtu prutového prvku v rovině $\gls{x}\gls{z}$ je pomocný vektor $\matr{v}$ rovnoběžný s globální osou \gls{Y}, v tom případě platí
\begin{equation}
    \begin{gathered}
        \matr{e}_{\gls{x}}
        = 
        \begin{Bmatrix}
            \frac{\Delta \gls{X}}{\gls{L}} &
            0 &
            \frac{\Delta \gls{Z}}{\gls{L}}
        \end{Bmatrix}^{\mathrm{T}}, 
        \\
        \matr{e}_{\gls{y}} 
        = 
        \begin{Bmatrix}
            0 & 1 & 0\\
        \end{Bmatrix}^{\mathrm{T}},
        \\
        \matr{e}_{\gls{z}}
        =
        \begin{Bmatrix}
            - \frac{\Delta \gls{Z}}{\gls{L}} &
            0 &
            \frac{\Delta \gls{X}}{\gls{L}}^{\mathrm{T}},
        \end{Bmatrix}
    \end{gathered}
\end{equation}
a transformační matice má tvar
\begin{equation}
    \gls{T}
    =
    \begin{bmatrix}
        \frac{\Delta \gls{X}}{\gls{L}} & 0 & \frac{\Delta \gls{Z}}{\gls{L}} \\
        0 & 1 & 0 \\
        - \frac{\Delta \gls{Z}}{\gls{L}} &
        0 &
        \frac{\Delta \gls{X}}{\gls{L}}
    \end{bmatrix}.
\end{equation}

Při ortogonální transformaci souřadnic je transformační matice \gls{T} ortogonální, to znamená, že
\begin{equation}
    \gls{T}^{-1} = \gls{T}^{\mathrm{T}}
\end{equation}

Pro vektor koncových posunutí a vektor zatížení lze podle vztahu \ref{eq:P_transformation} psát
\begin{equation} \label{eq:ltg}
    \begin{aligned}
        \gls{d}_{\ell} = \gls{T} \gls{d}_{g}, \\
        \gls{f}_{\ell} = \gls{T} \gls{f}_{g}.
    \end{aligned}
\end{equation}

Dosazením vztahu \ref{eq:ltg} do podmínek rovnováhy odvodíme vztah pro transformaci matice tuhosti z lokálního do souřadného systému,
\begin{equation}
    \begin{gathered}
        \gls{K}_{\ell} \gls{d}_{\ell} - \gls{f}_{\ell} = \matr{0} \\
        \gls{K}_{\ell} \gls{T} \gls{d}_{g} - \gls{T} \gls{f}_{g} = \matr{0} \\
        \underbrace{\gls{T}^\mathrm{T} \gls{K}_{\ell} \gls{T}}_{\gls{K}_g} \gls{d}_{g} - \gls{f}_{g} = \matr{0},
    \end{gathered}
\end{equation}
globální matici tuhosti vypočítáme podle vztahu
\begin{equation}
    \gls{K}_g = \gls{T}^\mathrm{T} \gls{K}_{\ell} \gls{T}.
\end{equation}