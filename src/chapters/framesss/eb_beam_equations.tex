\subsection{Ohýbaný prvek bez vlivu smyku} \label{sec:EB_beam}

\begin{figure}[H]
    \newcommand*{\xS}{0}
\newcommand*{\xE}{10}
\newcommand*{\wS}{3}
\newcommand*{\wE}{4.5}
%\newcommand*{\phiS}{0.3}
%\newcommand*{\phiE}{0.1}
\newcommand*{\h}{2}
\newcommand*{\xDivs}{20} % Number of divisions along the x-axis
\newcommand*{\yDivs}{4}  % Number of divisions along the y-axis

\begin{tikzpicture}[>={Stealth[inset=0pt,length=8pt,angle'=28,round]}, scale=0.8]
    % Draw axes
    \draw[->] (0, 0) -- (12, 0) node[above] {\gls{x}};
    \draw[->] (0, 0) -- (0, 5) node[above] {\gls{z}};
    
    % Draw the initial bar
    \draw[black] (\xS, -\h/2) rectangle (\xE, \h/2);

    \draw[dotted] (\xE, \h/2) -- (\xE, \wE);

    % Calculate spacing
    \pgfmathsetmacro{\dX}{(\xE-\xS)/(\xDivs)}
    \pgfmathsetmacro{\dY}{\h/(\yDivs)}
    \pgfmathsetmacro{\xDefDivs}{\xDivs-1}
    \pgfmathsetmacro{\yDefDivs}{\yDivs-1}
    \pgfmathsetmacro{\phiS}{(0.3-0.8)}
    \pgfmathsetmacro{\phiE}{(0.1-0.5)}
    %Draw beam grid
    \foreach \i in {1,...,\xDefDivs} {
        \draw[draw=black, dotted] (\xS + \i*\dX, -\h/2) -- (\xS + \i*\dX, \h/2); % Vertical grid lines
    }
    \foreach \i in {1,...,\yDefDivs} {
        \draw[draw=black, dotted] (\xS, -\h/2 + \i*\dY) -- (\xE, -\h/2 + \i*\dY); % Horizontal grid lines
    }

    \pgfmathsetmacro{\Len}{\xE - \xS}    % Beam centerline function

    \pgfmathdeclarefunction{n_1}{1}{%
        \pgfmathparse{2*(#1/\Len)^3 - 3*(#1/\Len)^2 + 1}%
    }

    \pgfmathdeclarefunction{n_2}{1}{%
        \pgfmathparse{#1*(1 - #1/\Len)^2}%
    }

    \pgfmathdeclarefunction{n_3}{1}{%
        \pgfmathparse{(3*(#1/\Len)^2 - 2*(#1/\Len)^3)}%
    }

    \pgfmathdeclarefunction{n_4}{1}{%
        \pgfmathparse{(#1^2 / \Len)*(#1/\Len - 1)}%
    }

    \pgfmathdeclarefunction{b_1}{1}{%
        \pgfmathparse{(6 * #1^2 / \Len^3 - 6 * #1 / \Len^2)}%
    }

    \pgfmathdeclarefunction{b_2}{1}{%
        \pgfmathparse{(4 * #1 / \Len - 3 * #1^2 / \Len^2 - 1)}%
    }

    \pgfmathdeclarefunction{b_3}{1}{%
        \pgfmathparse{(6*#1/\Len^2 - 6*#1^2/\Len^3)}%
    }

    \pgfmathdeclarefunction{b_4}{1}{%
        \pgfmathparse{(2 *#1/\Len - 3*#1^2/\Len^2)}%
    }

    \pgfmathdeclarefunction{w}{1}{%
        \pgfmathparse{
            (n_1(#1)*\wS + n_2(#1)*\phiS + n_3(#1)*\wE + n_4(#1)*\phiE)}%
    }

    \pgfmathdeclarefunction{phi}{1}{%
        \pgfmathparse{
            %(b_1(#1)*\wS + b_2(#1)*\phiS + b_3(#1)*\wE + b_4(#1)*\phiE)}%
            (b_2(#1)*\phiS - b_1(#1)*\wS - b_3(#1)*\wE + b_4(#1)*\phiE)}%
            %(b_2(#1)*\phiS + b_4(#1)*\phiE)}%
            }

    % vertical lines
    \foreach \i in {0,...,\xDivs} {
        \pgfmathsetmacro{\xM}{\xS + \i*\dX}
        \pgfmathsetmacro{\yM}{w(\xM)}

        \foreach \j in {1,...,\yDivs} {
            \pgfmathsetmacro{\k}{\j-1}
            \pgfmathsetmacro{\zB}{\j*\dY - \h/2}
            \pgfmathsetmacro{\zT}{\k*\dY - \h/2}

            \pgfmathsetmacro{\xA}{\xM + sin(deg(phi(\xM)))*\zB}
            \pgfmathsetmacro{\yA}{\yM + cos(deg(phi(\xM)))*\zB}
            \pgfmathsetmacro{\xB}{\xM + sin(deg(phi(\xM)))*\zT}
            \pgfmathsetmacro{\yB}{\yM + cos(deg(phi(\xM)))*\zT}

            \draw[ctulightblue] (\xA, \yA) -- (\xB, \yB);
        }
    }

    % horizontal lines
    \foreach \i in {0,...,\yDivs} {
        \pgfmathsetmacro{\z}{\i*\dY - \h/2}

        \foreach \j in {1,...,\xDivs} {
            \pgfmathsetmacro{\k}{\j-1}
            \pgfmathsetmacro{\xL}{\xS + \k*\dX}
            \pgfmathsetmacro{\xP}{\xS + \j*\dX}

            \pgfmathsetmacro{\xA}{\xL + sin(deg(phi(\xL)))*\z}
            \pgfmathsetmacro{\yA}{w(\xL) + cos(deg(phi(\xL)))*\z}
            \pgfmathsetmacro{\xB}{\xP + sin(deg(phi(\xP)))*\z}
            \pgfmathsetmacro{\yB}{w(\xP) + cos(deg(phi(\xP)))*\z}

            \draw[ctulightblue] (\xA, \yA) -- (\xB, \yB);
        }
    }

    \draw[color=ctublue, thick, samples=100, domain=\xS:\xE] plot (\x, {w(\x)});
    
    % nodes
    \draw[ultra thick] (\xS, 0) -- (\xE, 0);
    \draw[fill=white] (\xS, 0) circle (.25cm) node[text=black] {$a$};
    \draw[fill=white] (\xE, 0) circle (.25cm) node[text=black] {$b$};

    %\draw[|->|, draw=ctublue, text=ctublue] (\xS, +\h/2+0.3) -- node[above] {$u_a$} ++(\uStart, 0);
    %\draw[|->|, draw=ctublue, text=ctublue] (\xE, +\h/2+0.3) -- node[above] {$u_b$} ++(\uEnd, 0);
    %\draw[|->|, draw=ctublue, text=ctublue] (\xStart, +\height/2+0.3) -- node[above] {$u_a$} ++(\uStart, 0);

    \draw[draw=ctublue, |->|] (\xS - 0.5, 0.0) -- node[rotate=90, text=ctublue, above] {$w_a$} ++(0, \wS);
    \draw[draw=ctublue, |->|] (\xE + 0.8, 0.0) -- node[rotate=90, text=ctublue, above] {$w_b$} ++(0, \wE);

    \draw[|->|,domain=90:{deg(\phiS)+90}, draw=ctublue] plot ({1.5*cos(\x)}, {3+1.5*sin(\x)}) node[above right, text=ctublue] {$\varphi_a$};
    \draw[|->|,domain=270:{deg(\phiE)+270}, draw=ctublue] plot ({10+1.5*cos(\x)}, {4.5+1.5*sin(\x)}) node[below left, text=ctublue] {$\varphi_b$};

    \draw[|<->|, draw=black] (\xS, -\h/2-0.8) -- node[above] {\gls{L}} ++(\Len, 0);
    

\end{tikzpicture}
    \caption{Deformovaná konfigurace Euler-Bernoulliho nosníku}
    \label{fig:deformed_beam}
\end{figure}

Dle Euler-Bernoulliho hypotézy o zachování kolmosti příčných řezů k deformované střednici prutu lze posunutí \gls{u_i}[ ] ve směru osy \gls{x}, vyjádřit jako funkci vzdálenosti od osy \gls{z} a úhlu pootočení příčného řezu \gls{phi_i}[ ](\gls{x}) \cite[58]{ymkp}. Vektor posunutí \gls{u} tedy lze zapsat jako
\begin{equation}
    \gls{u} = 
    \begin{Bmatrix}
        \gls{u_i}[ ](\gls{x}, \gls{z}) \\
        \gls{w_i}[ ](\gls{x})
    \end{Bmatrix}
    =
    \begin{Bmatrix}
        -\gls{z} \pdv{\gls{w_i}[ ](\gls{x})}{\gls{x}} \\
        \gls{w_i}[ ](\gls{x})
    \end{Bmatrix}
    =
    \begin{Bmatrix}
        \gls{z} \gls{phi_i}[ ](\gls{x}) \\
        \gls{w_i}[ ](\gls{x})
    \end{Bmatrix}
\end{equation}

Vektor poměrných deformací \gls{eps} se díky přijatým předpokladům výrazně zjednodušší,
\begin{equation}
    \gls{eps}
    =
    \begin{Bmatrix}
        \gls{eps_i}[\gls{x}] \\
        \gls{gamma_i}[\gls{z}\gls{x}] \\
    \end{Bmatrix}
    =
    \begin{Bmatrix}
        \pdv{\gls{u_i}[ ]}{\gls{x}} \\
        \pdv{\gls{u_i}[ ]}{\gls{z}} + \pdv{\gls{w_i}[ ]}{\gls{x}} \\
    \end{Bmatrix}
    =
    \begin{Bmatrix}
        \pderivative{\gls{x}}\left( -\gls{z} \pdv{\gls{w_i}[ ]}{\gls{x}} \right) \\
        \pderivative{\gls{z}} \left( -\gls{z} \pdv{\gls{w_i}[ ]}{\gls{x}} \right) + \pdv{\gls{w_i}[ ]}{\gls{x}} \\
    \end{Bmatrix}.
\end{equation}

Nejprve upravíme člen \gls{gamma_i}[\gls{z}\gls{x}],
\begin{equation}
    \gls{gamma_i}[\gls{z}\gls{x}] 
    =
    \pderivative{\gls{z}} \left( -\gls{z} \pdv{\gls{w_i}[ ]}{\gls{x}} \right) + \pdv{\gls{w_i}[ ]}{\gls{x}}
    =
    -\pdv{\gls{w_i}[ ]}{\gls{x}} + \pdv{\gls{w_i}[ ]}{\gls{x}} 
    = 0,
\end{equation}
který je důsledkem Euler-Bernoulliho teorie nulový. Jedinou nenulovou složkou je díky přijatým předpokladům člen \gls{eps_i}[\gls{x}],
\begin{equation}
    \gls{eps_i}[\gls{x}](\gls{x}, \gls{z}) 
    = 
    \dv{\gls{x}} \left( -\gls{z} \dv{\gls{w_i}[ ](\gls{x})}{\gls{x}} \right) 
    =
    - \gls{z} \dv[2]{\gls{w_i}[ ](\gls{x})}{\gls{x}} 
    =
    \gls{z}\gls{kappa}(\gls{x}),
\end{equation}
kde záporně vzatou druhou derivaci průhybu \gls{w_i}[ ] podle \gls{x} označíme jako křivost \gls{kappa},
\begin{equation}
    \gls{kappa} = \dv{\gls{phi_i}[ ](\gls{x})}{\gls{x}} = - \dv[2]{\gls{w_i}[ ](\gls{x})}{\gls{x}}.
\end{equation}

Vektor napětí má také pouze jednu nenulovou složku,
\begin{equation}
    \gls{sigma_i}[\gls{x}](\gls{x}, \gls{z}) = \gls{E} \gls{eps_i}[\gls{x}] = \gls{E} \gls{kappa}(\gls{x}) \gls{z}.
\end{equation}

Funkci ohybového momentu lze získat integrací po průřezu podle vztahu,
\begin{equation}
    \gls{bending_moment}(\gls{x})
    = 
    \int_{\gls{A}} \gls{z} \gls{sigma_i}[\gls{x}](\gls{x}, \gls{z}) \dd{\gls{A}}
    =
    \gls{E}  \int_{\gls{A}} \gls{z}^2 \dd{\gls{A}} \gls{kappa}(\gls{x})
    =
    \gls{E} \gls{I_y} \gls{kappa}(\gls{x}).
\end{equation}

\subsubsection{Bázové funkce} \label{sec:eb_shape_functions}

Ohýbaný prut je v rovině dán pomocí dvou uzlů, $a$ a $b$. V každém koncovém uzlu zavedeme koncové posunutí ve směru lokální osy \gls{z} a pootočení okolo lokální osy \gls{y}, viz obr. \ref{fig:deformed_beam}. Vektor zobecněných posunutí má tvar,
\begin{equation}
    \gls{d} = \begin{Bmatrix}
        \gls{w_i}[a] \\ \gls{phi_i}[a] \\ \gls{w_i}[b] \\ \gls{phi_i}[b]
    \end{Bmatrix}
\end{equation}

Posunutí $\gls{w_i}[ ](\gls{x})$ budeme opět aproximovat pomocí vektoru koncových posunutí \gls{d} a matice bázových funckí \gls{N},
\begin{equation}
    \label{eq:deformation_approximation}
    \gls{w_i}[ ](\gls{x}) \approx \gls{N}(\gls{x}) \gls{d} = 
    \begin{bmatrix}
        \gls{N_i}[1](\gls{x}) &
        \gls{N_i}[2](\gls{x}) & 
        \gls{N_i}[3](\gls{x}) & 
        \gls{N_i}[4](\gls{x}) 
    \end{bmatrix}
    \begin{Bmatrix}
        \gls{w_i}[a] \\
        \gls{phi_i}[a] \\
        \gls{w_i}[b] \\
        \gls{phi_i}[b] \\
    \end{Bmatrix}.
\end{equation}

Pootočení průřezu \gls{phi_i}[ ](\gls{x}) lze určit ze vztahu,
\begin{equation}
    \label{eq:rotation_approximation}
    \gls{phi_i}[ ](\gls{x}) = - \dv{\gls{w_i}[ ](\gls{x})}{\gls{x}} \approx - \dv{\gls{N}(\gls{x})}{\gls{x}} \gls{d}.
\end{equation}

Bázové funkce \gls{N_i}[i] budeme hledat ve tvaru polynomu třetího stupně,
\begin{equation}
    \label{eq:eb_beam_shape_fnc}
    \gls{N_i}[i](\gls{x}) = a_{i} \gls{x}^{3} + b_{i} \gls{x}^{2} + c_{i} \gls{x} + d_{i},
\end{equation}
kde $a_i, b_i, c_i$ a $d_i$ jsou zatím neznámé koeficienty.

Derivace bázové funkce podle \gls{x} je rovna,
\begin{equation}
    \label{eq:eb_beam_shape_fnc_dx}
    \dv{\gls{N_i}[i](\gls{x})}{\gls{x}} = 3 a_{i} \gls{x}^{2} + 2 b_{i} \gls{x} + c_{i}.
\end{equation}


Funkce \ref{eq:eb_beam_shape_fnc} a \ref{eq:eb_beam_shape_fnc_dx} lze přepsat do maticové podoby,
\begin{equation}
    \label{eq:shape_function_coefficients_vector}
    \gls{N_i}[i](\gls{x}) = 
    \begin{bmatrix}
        \gls{x}^3 &
        \gls{x}^2 &
        \gls{x} &
        1
    \end{bmatrix}
    \begin{Bmatrix}
        a_i \\
        b_i \\
        c_i \\
        d_i \\
    \end{Bmatrix},
\end{equation}

\begin{equation}
    \label{eq:derivation_shape_function_coefficients_vector}
    \dv{\gls{N_i}[i](\gls{x})}{\gls{x}} = 
    \begin{bmatrix}
        3\gls{x}^2 &
        2\gls{x} &
        1 &
        0
    \end{bmatrix}
    \begin{Bmatrix}
        a_i \\
        b_i \\
        c_i \\
        d_i \\
    \end{Bmatrix},
\end{equation}

Dosazením vztahu \ref{eq:shape_function_coefficients_vector} do \ref{eq:deformation_approximation} získáme,
\begin{equation}
    \label{eq:w_midstep}
    \gls{w_i}[ ] (\gls{x}) = 
    \begin{bmatrix}
        \gls{x}^3 &
        \gls{x}^2 &
        \gls{x} &
        1
    \end{bmatrix}
    \begin{bmatrix}
        a_1 & a_2 & a_3 & a_4 \\
        b_1 & b_2 & b_3 & b_4 \\
        c_1 & c_2 & c_3 & c_4 \\
        d_1 & d_2 & d_3 & d_4 \\
    \end{bmatrix}
    \begin{Bmatrix}
        \gls{w_i}[a] \\
        \gls{phi_i}[a] \\
        \gls{w_i}[b] \\
        \gls{phi_i}[b] \\
    \end{Bmatrix},
\end{equation}
a dosazením \ref{eq:derivation_shape_function_coefficients_vector} do \ref{eq:rotation_approximation},
\begin{equation}
    \label{eq:phi_midstep}
    \gls{phi_i}[ ] (\gls{x}) = 
    \begin{bmatrix}
        - 3\gls{x}^2 &
        - 2\gls{x} &
        - 1 &
        0
    \end{bmatrix}
    \begin{bmatrix}
        a_1 & a_2 & a_3 & a_4 \\
        b_1 & b_2 & b_3 & b_4 \\
        c_1 & c_2 & c_3 & c_4 \\
        d_1 & d_2 & d_3 & d_4 \\
    \end{bmatrix}
    \begin{Bmatrix}
        \gls{w_i}[a] \\
        \gls{phi_i}[a] \\
        \gls{w_i}[b] \\
        \gls{phi_i}[b] \\
    \end{Bmatrix}.
\end{equation}
Okrajové podmínky, patrné z \autoref{fig:deformed_beam} na prvku jsou,
\begin{subequations}
    \begin{equation}
        \gls{w_i}[ ](0) = \gls{w_i}[a],
    \end{equation}
    \begin{equation}
        \gls{phi_i}[ ](0) = \gls{phi_i}[a],
    \end{equation}
    \begin{equation}
        \gls{w_i}[ ](\gls{L}) = \gls{w_i}[b],
    \end{equation}
    \begin{equation}
        \gls{phi_i}[ ](\gls{L}) = \gls{phi_i}[b].
    \end{equation}
\end{subequations}

Dosazením okrajových podmínek do rovnic \ref{eq:w_midstep} a \ref{eq:phi_midstep} dostaneme soustavu rovnic,
\begin{equation} \label{eq:eb_beam_A}
    \underbrace{
    \begin{bmatrix}
        0 & 0 & 0 & 1 \\
        0 & 0 & -1 & 0 \\
        \gls{L}^{3} & \gls{L}^{2} & \gls{L} & 1 \\
        -3 \gls{L}^{2} & -2 \gls{L} & -1 & 0 \\
    \end{bmatrix}}_{\matr{A}}
    \begin{bmatrix}
        a_1 & a_2 & a_3 & a_4 \\
        b_1 & b_2 & b_3 & b_4 \\
        c_1 & c_2 & c_3 & c_4 \\
        d_1 & d_2 & d_3 & d_4 \\
    \end{bmatrix}
    \begin{Bmatrix}
        \gls{w_i}[a] \\
        \gls{phi_i}[a] \\
        \gls{w_i}[b] \\
        \gls{phi_i}[b] \\
    \end{Bmatrix}
    =
    \begin{Bmatrix}
        \gls{w_i}[a] \\
        \gls{phi_i}[a] \\
        \gls{w_i}[b] \\
        \gls{phi_i}[b] \\
    \end{Bmatrix},
\end{equation}
přenásobením zleva maticí $\matr{A}^{-1}$ dostaneme,
\begin{equation}
    \label{eq:shape_func_bnd_cond}
    \begin{bmatrix}
        a_1 & a_2 & a_3 & a_4 \\
        b_1 & b_2 & b_3 & b_4 \\
        c_1 & c_2 & c_3 & c_4 \\
        d_1 & d_2 & d_3 & d_4 \\
    \end{bmatrix}
    \begin{Bmatrix}
        \gls{w_i}[a] \\
        \gls{phi_i}[a] \\
        \gls{w_i}[b] \\
        \gls{phi_i}[b] \\
    \end{Bmatrix}
    =
    \begin{bmatrix}
        0 & 0 & 0 & 1 \\
        0 & 0 & -1 & 0 \\
        \gls{L}^{3} & \gls{L}^{2} & \gls{L} & 1 \\
        -3 \gls{L}^{2} & -2 \gls{L} & -1 & 0 \\
    \end{bmatrix}^{-1}
    \begin{Bmatrix}
        \gls{w_i}[a] \\
        \gls{phi_i}[a] \\
        \gls{w_i}[b] \\
        \gls{phi_i}[b] \\
    \end{Bmatrix}.
\end{equation}

Pro další výpočty použijeme knihovnu SymPy\footnote{
    SymPy je open source knihovna pro symbolické výpočty v jazyce Python. Umožňuje manipulaci a řešení matematických výrazů v symbolické podobě, což zahrnuje algebraické operace, diferenciální a integrální počty, řešení rovnic, práci s maticemi a další matematické úlohy \cite{sympy}.
} v prostředí JupyterLab\footnote{
JupyterLab je interaktivní vývojové prostředí, které umožňuje vytváření a sdílení dokumentů obsahujících živý kód, rovnice, vizualizace a text. Je navrženo jako nástupce klasických Jupyter Notebooků a nabízí rozšířené možnosti pro práci s daty a vývoj aplikací. JupyterLab podporuje různé programovací jazyky, přičemž nejpoužívanější je Python. Uživatelé mohou využívat JupyterLab k analýze dat, strojovému učení, vizualizaci dat a vědeckému výzkumu, díky jeho flexibilitě a širokému spektru dostupných rozšíření a integrací.
}.

Nejprve importujeme knihovnu SymPy do prostředí JupyterLab.

\begin{tcolorbox}[breakable, size=fbox, boxrule=1pt, pad at break*=1mm,colback=cellbackground, colframe=cellborder]
    \prompt{In}{incolor}{1}{\boxspacing}
    \begin{Verbatim}[commandchars=\\\{\}]
    \PY{k+kn}{import} \PY{n+nn}{sympy} \PY{k}{as} \PY{n+nn}{smp}
    \end{Verbatim}
\end{tcolorbox}

V dalším kroku definujeme symbolické proměnné, které budeme potřebovat pro další výpočet.

\begin{tcolorbox}[breakable, size=fbox, boxrule=1pt, pad at break*=1mm,colback=cellbackground, colframe=cellborder]
    \prompt{In}{incolor}{2}{\boxspacing}
    \begin{Verbatim}[commandchars=\\\{\}]
    \PY{n}{L} \PY{o}{=} \PY{n}{smp}\PY{o}{.}\PY{n}{symbols}\PY{p}{(}\PY{l+s+s1}{\PYZsq{}}\PY{l+s+s1}{L}\PY{l+s+s1}{\PYZsq{}}\PY{p}{)}
    \PY{n}{w\PYZus{}a}\PY{p}{,} \PY{n}{phi\PYZus{}a}\PY{p}{,} \PY{n}{w\PYZus{}b}\PY{p}{,} \PY{n}{phi\PYZus{}b} \PY{o}{=} \PY{n}{smp}\PY{o}{.}\PY{n}{symbols}\PY{p}{(}
        \PY{l+s+s1}{\PYZsq{}}\PY{l+s+s1}{w\PYZus{}a varphi\PYZus{}a w\PYZus{}b varphi\PYZus{}b}\PY{l+s+s1}{\PYZsq{}}
    \PY{p}{)}
    \PY{n}{x} \PY{o}{=} \PY{n}{smp}\PY{o}{.}\PY{n}{symbols}\PY{p}{(}\PY{l+s+s1}{\PYZsq{}}\PY{l+s+s1}{x}\PY{l+s+s1}{\PYZsq{}}\PY{p}{)}
    \PY{n}{EI} \PY{o}{=} \PY{n}{smp}\PY{o}{.}\PY{n}{symbols}\PY{p}{(}\PY{l+s+s1}{\PYZsq{}}\PY{l+s+s1}{EI\PYZus{}y}\PY{l+s+s1}{\PYZsq{}}\PY{p}{)}
    \end{Verbatim}
\end{tcolorbox}
        
Do proměnné \texttt{A} uložíme matici $\matr{A}$ z rovnice \ref{eq:eb_beam_A} a vypíšeme ji do výstupní buňky.

\begin{tcolorbox}[breakable, size=fbox, boxrule=1pt, pad at break*=1mm,colback=cellbackground, colframe=cellborder]
    \prompt{In}{incolor}{3}{\boxspacing}
    \begin{Verbatim}[commandchars=\\\{\}]
    \PY{n}{A} \PY{o}{=} \PY{n}{smp}\PY{o}{.}\PY{n}{Matrix}\PY{p}{(}
        \PY{p}{[}
            \PY{p}{[}\PY{l+m+mi}{0}\PY{p}{,} \PY{l+m+mi}{0}\PY{p}{,} \PY{l+m+mi}{0}\PY{p}{,} \PY{l+m+mi}{1}\PY{p}{]}\PY{p}{,}
            \PY{p}{[}\PY{l+m+mi}{0}\PY{p}{,} \PY{l+m+mi}{0}\PY{p}{,} \PY{o}{\PYZhy{}}\PY{l+m+mi}{1}\PY{p}{,} \PY{l+m+mi}{0}\PY{p}{]}\PY{p}{,}
            \PY{p}{[}\PY{n}{L}\PY{o}{*}\PY{o}{*}\PY{l+m+mi}{3}\PY{p}{,} \PY{n}{L}\PY{o}{*}\PY{o}{*}\PY{l+m+mi}{2}\PY{p}{,} \PY{n}{L}\PY{p}{,} \PY{l+m+mi}{1}\PY{p}{]}\PY{p}{,}
            \PY{p}{[}\PY{o}{\PYZhy{}}\PY{l+m+mi}{3}\PY{o}{*}\PY{n}{L}\PY{o}{*}\PY{o}{*}\PY{l+m+mi}{2}\PY{p}{,} \PY{o}{\PYZhy{}}\PY{l+m+mi}{2}\PY{o}{*}\PY{n}{L}\PY{p}{,} \PY{o}{\PYZhy{}}\PY{l+m+mi}{1}\PY{p}{,} \PY{l+m+mi}{0}\PY{p}{]}
        \PY{p}{]}
    \PY{p}{)}
    
    \PY{n}{A}
    \end{Verbatim}
\end{tcolorbox}
   
\newpage
                
\prompt{Out}{outcolor}{3}{}
    
    $\displaystyle \left[\begin{matrix}0 & 0 & 0 & 1\\0 & 0 & -1 & 0\\\gls{L}^{3} & \gls{L}^{2} & \gls{L} & 1\\- 3 \gls{L}^{2} & - 2 \gls{L} & -1 & 0\end{matrix}\right]$

\vspace{0.3cm}

Matici bázových funkcí \gls{N} určíme podle vztahu \ref{eq:shape_function_coefficients_vector}.

\begin{tcolorbox}[breakable, size=fbox, boxrule=1pt, pad at break*=1mm,colback=cellbackground, colframe=cellborder]
    \prompt{In}{incolor}{4}{\boxspacing}
    \begin{Verbatim}[commandchars=\\\{\}]
    \PY{n}{N} \PY{o}{=} \PY{n}{smp}\PY{o}{.}\PY{n}{Matrix}\PY{p}{(}\PY{p}{[}\PY{p}{[}\PY{n}{x}\PY{o}{*}\PY{o}{*}\PY{l+m+mi}{3}\PY{p}{,} \PY{n}{x}\PY{o}{*}\PY{o}{*}\PY{l+m+mi}{2}\PY{p}{,} \PY{n}{x}\PY{p}{,} \PY{l+m+mi}{1}\PY{p}{]}\PY{p}{]}\PY{p}{)} \PY{o}{@} \PY{n}{A}\PY{o}{.}\PY{n}{inv}\PY{p}{(}\PY{p}{)}
    \PY{n}{N}
    \end{Verbatim}
\end{tcolorbox}
     
                
\prompt{Out}{outcolor}{4}{}
    
    $\displaystyle \left[\begin{matrix}1 - \frac{3 \gls{x}^{2}}{\gls{L}^{2}} + \frac{2 \gls{x}^{3}}{\gls{L}^{3}} & - \gls{x} + \frac{2 \gls{x}^{2}}{\gls{L}} - \frac{\gls{x}^{3}}{\gls{L}^{2}} & \frac{3 \gls{x}^{2}}{\gls{L}^{2}} - \frac{2 \gls{x}^{3}}{\gls{L}^{3}} & \frac{\gls{x}^{2}}{\gls{L}} - \frac{\gls{x}^{3}}{\gls{L}^{2}}\end{matrix}\right]$

\vspace{0.3cm}

Po zjednodušení vyjde,

\begin{align}
    \gls{N_i}[1] (\gls{x}) &= 2 \left( \frac{\gls{x}}{\gls{L}} \right)^3 - 3 \left( \frac{\gls{x}}{\gls{L}} \right)^2 + 1, \\
    \gls{N_i}[2] (\gls{x}) & = - \gls{x} \left( \frac{\gls{x}}{\gls{L}} - 1\right)^2, \\
    \gls{N_i}[3] (\gls{x}) &= - 2 \left( \frac{\gls{x}}{\gls{L}} \right)^3 + 3 \left( \frac{\gls{x}}{\gls{L}} \right)^2, \\
    \gls{N_i}[4] (\gls{x}) &= \frac{\gls{x}}{\gls{L}} \left( 1 - \frac{\gls{x}}{\gls{L}} \right).
\end{align}
                
\begin{figure}[H]
    \begin{tikzpicture}[>={Stealth[inset=0pt,length=8pt,angle'=28,round]},]
    % N1
    \draw[draw=black] (0.0, 0.0) node[below] {0} -- (5.0, 0.0) node[below] {\gls{L}};
    \draw[color=ctublue, domain=0:5, variable=\x, line width=0.4mm] plot ({\x}, {(2/5^3)*\x^3 - (3/5^2)*\x^2 + 1});
    \draw[->] (0, 0) -- (0, 1);
    \node[rotate=90, text=ctublue, above] at (0, 0.5) {$1$};
    \node[text=ctublue, above] at (2.5, 0.5) {$N_1(x)$};

    %N2
    \draw[draw=black] (6.0, 0.0) node[above] {0} -- (11.0, 0.0) node[above] {\gls{L}};
    \draw[color=ctublue, domain=0:5, variable=\x, line width=0.4mm] plot ({\x+6}, {-\x * ((\x/5 - 1))^2});
    \draw[] (6, 0) -- (7, -1);
    \draw[->,domain=0:-45] plot ({6+cos(\x)}, {sin(\x)}) node[below left, text=ctublue] {$1$};
    \node[text=ctublue, below right] at (8.5, -0.5) {$N_2(x)$};

    %N3
    \draw[draw=black] (0.0, -3.0) node[below] {0} -- (5.0, -3.0) node[below] {\gls{L}};
    \draw[color=ctublue, domain=0:5, variable=\x, line width=0.4mm] plot ({\x}, {-3 + (-2*(\x/5)^3 + 3*(\x/5)^2)});
    \draw[->] (5, -3) -- (5, -2);
    \node[rotate=90, text=ctublue, above] at (5, -2.5) {$1$};
    \node[text=ctublue, above] at (2.5, -2.5) {$N_3(x)$};

    %N4
    \draw[draw=black] (6.0, -3.0) node[below] {0} -- (11.0, -3.0) node[below] {\gls{L}};
    \draw[color=ctublue, domain=0:5, variable=\x, line width=0.4mm] plot ({\x+6}, {-3+(\x^2/5 * (1-\x/5))});
    \draw[] (11, -3) -- (10, -2);
    \draw[->,domain=180:135] plot ({11+cos(\x)}, {-3+sin(\x)}) node[above right, text=ctublue] {$1$};
    \node[text=ctublue, above left] at (8.5, -2.5) {$N_4(x)$};

\end{tikzpicture}
    \caption{Bázové funkce ohýbaného nosníku}
    \label{fig:N_1}
\end{figure}


Dále určíme matici \gls{B}. Vyjdeme ze vztahu mezi křivostí a průhybem,
\begin{equation}
    \gls{kappa} = - \dv[2]{\gls{w_i}[ ](\gls{x})}{\gls{x}} \approx  - \dv[2]{\gls{N}(\gls{x})}{\gls{x}} \gls{d} = \gls{B} \gls{d}.
\end{equation}

Matici \gls{B}, popisující vztah mezi průhybem a přetvořením, vypočítáme následovně,
\begin{equation}
    \gls{B} = - \dv[2]{\gls{N}(\gls{x})}{\gls{x}}.
\end{equation}

    
\begin{tcolorbox}[breakable, size=fbox, boxrule=1pt, pad at break*=1mm,colback=cellbackground, colframe=cellborder]
    \prompt{In}{incolor}{5}{\boxspacing}
    \begin{Verbatim}[commandchars=\\\{\}]
    \PY{n}{B} \PY{o}{=} \PY{o}{\PYZhy{}}\PY{n}{N}\PY{o}{.}\PY{n}{diff}\PY{p}{(}\PY{n}{x}\PY{p}{,}\PY{n}{x}\PY{p}{)}
    \PY{n}{B}\PY{o}{.}\PY{n}{applyfunc}\PY{p}{(}\PY{n}{smp}\PY{o}{.}\PY{n}{simplify}\PY{p}{)}
    \end{Verbatim}
\end{tcolorbox}
                
\prompt{Out}{outcolor}{5}{}
        
    $\displaystyle \left[\begin{matrix}\frac{6 \left(L - 2 x\right)}{L^{3}} & \frac{2 \left(- 2 L + 3 x\right)}{L^{2}} & \frac{6 \left(- L + 2 x\right)}{L^{3}} & \frac{2 \left(- L + 3 x\right)}{L^{2}}\end{matrix}\right]$
    
\vspace{0.5cm}

\subsubsection{Matice tuhosti} \label{sec:K_eb_beam}

Matici tuhosti vypočítáme opět podle vztahu \ref{eq:K}, uvažujeme konstantní ohybovou tuhost průřezu \gls{E}\gls{I_y} na celém intervalu.

\begin{tcolorbox}[breakable, size=fbox, boxrule=1pt, pad at break*=1mm,colback=cellbackground, colframe=cellborder]
    \prompt{In}{incolor}{6}{\boxspacing}
    \begin{Verbatim}[commandchars=\\\{\}]
    \PY{n}{K} \PY{o}{=} \PY{n}{EI} \PY{o}{*} \PY{n}{smp}\PY{o}{.}\PY{n}{integrate}\PY{p}{(}
        \PY{n}{B}\PY{o}{.}\PY{n}{transpose}\PY{p}{(}\PY{p}{)} \PY{o}{@} \PY{n}{B}\PY{p}{,} 
        \PY{p}{(}\PY{n}{x}\PY{p}{,} \PY{l+m+mi}{0}\PY{p}{,} \PY{n}{L}\PY{p}{)}
    \PY{p}{)}
    \PY{n}{K}
    \end{Verbatim}
\end{tcolorbox}
     
\prompt{Out}{outcolor}{6}{}
    
    $\displaystyle \left[\begin{matrix}\frac{12 \gls{E}\gls{I_y}}{\gls{L}^{3}} & - \frac{6 \gls{E}\gls{I_y}}{\gls{L}^{2}} & - \frac{12 \gls{E}\gls{I_y}}{\gls{L}^{3}} & - \frac{6 \gls{E}\gls{I_y}}{\gls{L}^{2}}\\- \frac{6 \gls{E}\gls{I_y}}{\gls{L}^{2}} & \frac{4 \gls{E}\gls{I_y}}{\gls{L}} & \frac{6 \gls{E}\gls{I_y}}{\gls{L}^{2}} & \frac{2 \gls{E}\gls{I_y}}{\gls{L}}\\- \frac{12 \gls{E}\gls{I_y}}{\gls{L}^{3}} & \frac{6 \gls{E}\gls{I_y}}{\gls{L}^{2}} & \frac{12 \gls{E}\gls{I_y}}{\gls{L}^{3}} & \frac{6 \gls{E}\gls{I_y}}{\gls{L}^{2}}\\- \frac{6 \gls{E}\gls{I_y}}{\gls{L}^{2}} & \frac{2 \gls{E}\gls{I_y}}{\gls{L}} & \frac{6 \gls{E}\gls{I_y}}{\gls{L}^{2}} & \frac{4 \gls{E}\gls{I_y}}{\gls{L}}\end{matrix}\right]$
    
\subsubsection{Vektor zatížení}

Stejně, jako v případě taženého-tlačeného prutu, uvažujeme lineární spojité zatížení působící po celé délce prvku,

\begin{equation}
    \overline{f_{\gls{z}}}(\gls{x}) = \frac{\overline{f}_b - \overline{f}_a}{\gls{L}} \gls{x} + \overline{f_a}.
\end{equation}
\begin{figure}[H]
    \newcommand*{\xStart}{0}
\newcommand*{\xEnd}{8}
\newcommand*{\height}{2}
\newcommand*{\offset}{0.35}
\newcommand*{\fStart}{1}
\newcommand*{\fEnd}{2.5}
\newcommand*{\xDivisions}{10} % Number of divisions along the x-axis

\begin{tikzpicture}[>={Stealth[inset=0pt,length=8pt,angle'=28,round]},]
    % Draw axes
    \draw[->] (0, 0) -- (9.5, 0) node[above] {\gls{x}};
    \draw[->] (0, 0) -- (0, -1.5) node[below] {\gls{z}};
    
    \draw[|-|, ultra thick] (\xStart, 0) -- (\xEnd, 0);
    \draw[fill=white] (\xStart-\offset, -\offset) circle (.25cm) node[text=black] {$a$};
    \draw[fill=white] (\xEnd+\offset, -\offset) circle (.25cm) node[text=black] {$b$};

    % Calculate spacing
    \pgfmathsetmacro{\xSpacing}{(\xEnd-\xStart)/\xDivisions}
    \pgfmathsetmacro{\Len}{(\xEnd-\xStart)}
    \pgfmathsetmacro{\slope}{(\fEnd - \fStart) / \Len}
    %Draw bar grid
    \foreach \i in {0,...,\xDivisions} {
        \pgfmathsetmacro{\x}{\xStart + \i*\xSpacing}

        %\draw[<-, draw=ctublue, thick] (\x, 0) -- (\x, \x * (\fEnd - \fStart) / \Len + \fStart); % Arrows along the bar
        \draw[<-, draw=ctublue, thick] (\x, 0) -- (\x, \x * \slope + \fStart);
    }   
    \draw[draw=ctublue, thick] (\xStart, 0) -- (\xStart, \fStart) -- node[above, text=ctublue] {$\overline{f_{z}}(x)$} (\xEnd, \fEnd) -- (\xEnd, 0);

    \draw[|->|, draw=ctublue] (-0.5, 0) -- node[rotate=90, text=ctublue, above] {$\overline{f}_a$} (-0.5, \fStart);
    \draw[|->|, draw=ctublue] (\xEnd+0.7, 0) -- node[rotate=90, text=ctublue, above] {$\overline{f}_b$} (\xEnd+0.7, \fEnd);
    \draw[|<->|, draw=black] (\xStart, -0.7) -- node[above] {\gls{L}} (\xEnd, -0.7);
\end{tikzpicture}
    \caption{Zatížení Euler-Bernoulliho ohýbaného nosníku}
    \label{fig:beam_load}
\end{figure}        

Integrací výrazu $\gls{N}^\mathrm{T} \overline{f}_z$ po délce prutu obdržíme výraz pro vektor zatížení.

\begin{tcolorbox}[breakable, size=fbox, boxrule=1pt, pad at break*=1mm,colback=cellbackground, colframe=cellborder]
    \prompt{In}{incolor}{7}{\boxspacing}
    \begin{Verbatim}[commandchars=\\\{\}]
    \PY{n}{f\PYZus{}a}\PY{p}{,} \PY{n}{f\PYZus{}b} \PY{o}{=} \PY{n}{smp}\PY{o}{.}\PY{n}{symbols}\PY{p}{(}\PY{l+s+s1}{\PYZsq{}}\PY{l+s+s1}{f\PYZus{}a f\PYZus{}b}\PY{l+s+s1}{\PYZsq{}}\PY{p}{)}
    
    \PY{n}{f\PYZus{}z} \PY{o}{=} \PY{n}{x} \PY{o}{*} \PY{p}{(}\PY{n}{f\PYZus{}b} \PY{o}{\PYZhy{}} \PY{n}{f\PYZus{}a}\PY{p}{)} \PY{o}{/} \PY{n}{L} \PY{o}{+} \PY{n}{f\PYZus{}a}
    
    \PY{n}{f} \PY{o}{=} \PY{n}{smp}\PY{o}{.}\PY{n}{integrate}\PY{p}{(}
        \PY{n}{N}\PY{o}{.}\PY{n}{transpose}\PY{p}{(}\PY{p}{)} \PY{o}{*} \PY{n}{f\PYZus{}z}\PY{p}{,}
        \PY{p}{(}\PY{n}{x}\PY{p}{,} \PY{l+m+mi}{0}\PY{p}{,} \PY{n}{L}\PY{p}{)}
    \PY{p}{)}
    \PY{n}{f}\PY{o}{.}\PY{n}{simplify}\PY{p}{(}\PY{p}{)}
    \end{Verbatim}
\end{tcolorbox}
     
                
\prompt{Out}{outcolor}{7}{}
        
    $\displaystyle \left[\begin{matrix}\frac{\gls{L} \left(7 f_{a} + 3 f_{b}\right)}{20}\\\gls{L}^{2} \left(- \frac{f_{a}}{20} - \frac{f_{b}}{30}\right)\\\frac{\gls{L} \left(3 f_{a} + 7 f_{b}\right)}{20}\\\gls{L}^{2} \left(\frac{f_{a}}{30} + \frac{f_{b}}{20}\right)\end{matrix}\right]$

