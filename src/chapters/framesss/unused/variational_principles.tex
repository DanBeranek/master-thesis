\section{Variační principy}

Následující definice jsou převzaty z publikace \cite[kap. 6.2]{ymkp}.

\begin{definition}[Lagrangeův variační princip]
    Při variaci složek posunutí, splňujících geometrické podmínky uvnitř oblasti \gls{omega} i na její hranici \gls{gamma}, je variace potenciální energie systému nulová.
\end{definition}

Důsledkem Lagrangeova variačního principu je věta o minimu potenciální energie systému.

\begin{definition}[Věta o minimu potenciální energie]
    Ze všech kinematicky přípustných stavů \gls{u}(\gls{x}) (t.j. stavů které vyhovují okrajovým podmínkám) nastane ten, pro který nabývá potenciální energie \gls{PI} minimální hodnoty.
\end{definition}

\begin{equation}
    \gls{delta}\gls{PI} = \gls{delta}\gls{U} - \gls{delta}\gls{W} = 0,
\end{equation}


\subsection{Vnitřní deformační energie}
Pro hustotu deformační energie \gls{Phi}, schematicky naznačenou na obr. \ref{fig:stress_strain}, platí vztah

\begin{equation}
    \gls{sigma} = \pdv{\gls{Phi}}{\gls{eps}},
\end{equation}

\begin{figure}[H]
    \begin{tikzpicture}[>={Stealth[inset=0pt,length=6pt,angle'=28,round]}]

    \draw[draw=white, domain=0:5, variable=\x, thick, fill=ctublue!10] plot ({\x}, {-0.1 * (\x - 6.325)^2 + 4}) -- (5, 0) -- cycle;
    \draw[draw=ctublue, domain=0:5, variable=\x, thick] plot ({\x}, {-0.1 * (\x - 6.325)^2 + 4});

    \draw[draw=black, fill=ctulightblue, thick] (5,0) -- (5.5, 0) -- (5.5,3.932) -- (5,3.824) -- cycle;

    \draw[->, thick] (0, 0) -- (6.5, 0) node[above] {$\gls{eps_i}[ ]$};
    \draw[->, thick] (0, 0) -- (0, 4.5) node[left] {$\gls{sigma_i}[ ]$};

    \draw[|<->|, draw=black] (5.0, -0.7) -- node[above] {$\dd{\gls{eps_i}[ ]}$} ++(0.5, 0);

    \node[] at (3.5, 1.5) {\gls{Phi}};
    \coordinate (A) at (5.17, 1.939);
    \node[right=0.7 cm of A] (B) {$\dd{\gls{Phi}}$};
    \draw[*-] (A) -- (B);
\end{tikzpicture}
    \caption{Deformační potenciál}
    \label{fig:stress_strain}
\end{figure}

Vnitřní deformační energii lze vyjádřit vztahem

\begin{equation}
    \gls{U}(\gls{u}) = \int_{\gls{omega}} \gls{Phi} \dd{\gls{omega}} = \int_{\gls{omega}} \left( \int_{0}^{\gls{eps_i}[ ]} \gls{sigma}^\mathrm{T} \dd{\gls{eps}}   \right) \dd{\gls{omega}}.
\end{equation}

Pro lineárně pružný material je zřejmé, že vnitřní deformační energie bude

\begin{equation}
    \gls{U}(\gls{u}) = \frac{1}{2} \int_{\gls{omega}} {\gls{eps}}^\mathrm{T} \gls{sigma} \dd{\gls{omega}}.
\end{equation}

\subsection{Vnější deformační energie}

Z Lagrangeova variačního principu minima potenciální energie, který vychází z principu virtuální práce, vyplývá deformační metoda

