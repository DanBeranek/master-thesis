\section{Deformační metoda}




\subsection{Základní rovnice}

Deformační metoda je metoda pro výpočet neznámých styčníkových posunů a pootočení staticky určitých i neurčitých prutových konstrukcí. Soustavu základních rovnic, vyjadřujících vztah mezi působícím zatížením a deformací konstrukce, lze maticově vyjádřit ve tvaru

\begin{equation}
    \label{eq:Kuf}
    \gls{K} \gls{u} = \gls{f},
\end{equation}

kde \gls{K} je matice tuhosti konstrukce, \gls{u} je vektor posunutí a \gls{f} je vektor zatížení.

\subsection{Matice tuhosti}

Matice tuhosti \gls{K} při lineárně pružném výpočtu je symetrická čtvercová matice o rozměru $\gls{n_dof} \times \gls{n_dof}$, kde \gls{n_dof} značí celkový počet stupňů volnosti. 


