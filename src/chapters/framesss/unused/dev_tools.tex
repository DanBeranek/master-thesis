Pro usnadnění vývoje a zajištění kvality kódu byly použity následující nástroje:
\begin{itemize}
    \item \textbf{pre-commit} \cite{precommit}: Nástroj pro automatizaci úloh před commitováním kódu, který zajišťuje, že všechny změny projdou definovanými kontrolami.
    \item \textbf{Nox} \cite{nox}: Automatizační nástroj pro správu testovacích prostředí a spouštění úloh.
    \item \textbf{Ruff} \cite{ruff}: Rychlý linter pro Python, který pomáhá udržovat konzistenci kódu a detekovat chyby.
    \item \textbf{Poetry} \cite{poetry}: Nástroj pro správu závislostí a balíčkování, který zjednodušuje správu projektů v Pythonu.
    \item \textbf{Black} \cite{black}: Formátovač kódu, který automaticky formátuje kód podle stanovených pravidel.
    \item \textbf{mypy} \cite{mypy}:  Statický typový kontroler pro Python, který pomáhá detekovat potenciální chyby v kódu.
    \item \textbf{pytest} \cite{pytest}: Framework pro psaní a spouštění testů.
    \item \textbf{Sphinx} \cite{sphinx}: Nástroj pro generování dokumentace, který umožňuje snadnou tvorbu a správu dokumentace kódu.
    \item \textbf{SonarCloud} \cite{sonarcloud}: Platforma pro analýzu kvality kódu, která poskytuje nástroje pro kontinuální integraci a hodnocení kódu.
\end{itemize}