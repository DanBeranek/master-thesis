\section{Princip virtuálních prací}

Princip virtuálních prací a variační principy mechaniky jsou základem většiny přibližných metod mechaniky. Princip virtuálních prací lze dělit na princip virtuálních posunutí a princip virtuálních sil.

Publikace \cite[53]{prpe20} uvádí následující definice:
\begin{definition}[Virtuální posun]
    \label{def:virtual_displacement}
    Libovolný možný posun elementu mechanické soustavy, který je v~souladu s~jejími pohybovými možnostmi. Elementem mechanické soustavy rozumíme jakoukoliv její část, jež je s~ostatními částmi spojena vazbami.
\end{definition}

\begin{definition}[Virtuální deformace]
    Jsou odvozeny z~virtuálních posunů pomocí geometrických rovnic. Virtuální posuny i deformace nenarušují vazby soustavy. Jsou fiktivní, myšlené a uděleme je elementům soustavy bez ohledu na síly a napětí, které na ně působí.
\end{definition}

\begin{definition}[Virtuální síla]
    Duální protějšek virtuálního posunu. Je to síla, kterou na soustavu umisťujeme nezávisle na skutečných posunech.
\end{definition}

\begin{definition}[Virtuální napětí]
    Fiktivní, myšlené veličiny. Jsou stanoveny tak, aby v~každém bodě tělesa, které je zatíženo virtuálními silami, byla zajištěna rovnováha.
\end{definition}

Předpokládejme, že se těleso nachází v~jisté rovnovážné konfiguraci, která je jednoznačně popsána vektorovým polem posunutí $$\gls{u} = \begin{Bmatrix}
    \gls{u_i}[ ] & \gls{v_i}[ ] & \gls{w_i}[ ]
\end{Bmatrix}^\mathrm{T},$$ tenzorovým polem deformace $$\gls{eps} = \begin{Bmatrix}
    \gls{eps_i}[\gls{x}] & \gls{eps_i}[\gls{y}] & \gls{eps_i}[\gls{z}] & \gls{gamma_i}[\gls{y}\gls{z}] & \gls{gamma_i}[\gls{z}\gls{x}] & \gls{gamma_i}[\gls{x}\gls{y}]
\end{Bmatrix}^\mathrm{T},$$ a tenzorovým polem napětí $$\gls{sigma} = \begin{Bmatrix}
    \gls{sigma_i}[\gls{x}] & \gls{sigma_i}[\gls{y}] & \gls{sigma_i}[\gls{z}] & \gls{tau_i}[\gls{y}\gls{z}] & \gls{tau_i}[\gls{z}\gls{x}] & \gls{tau_i}[\gls{x}\gls{y}]
\end{Bmatrix}^\mathrm{T}.$$

Dále předpokládejme, že každému bodu tělesa je udělen malý virtuální posun 
$$\gls{delta}\gls{u} = \begin{Bmatrix}
    \gls{delta}\gls{u_i}[ ] & \gls{delta}\gls{v_i}[ ] & \gls{delta}\gls{w_i}[ ]
\end{Bmatrix}^\mathrm{T},$$
který jej vychýlí z~původní polohy v~rovnovážné konfiguraci. Funkce \gls{delta}\gls{u_i}[ ], \gls{delta}\gls{v_i}[ ] a \gls{delta}\gls{w_i}[ ] jsou ve smyslu definice \ref{def:virtual_displacement} spojité a mají spojité první parciální derivace podle proměnných \gls{x}, \gls{y} a \gls{z}. Výsledné posuny $\left( \gls{u} + \gls{delta}\gls{u} \right)$ a odvozené deformace $\left( \gls{eps} + \gls{delta}\gls{eps} \right)$ se nazývají kinematicky přípustné.

Připojením virtuálních objemových sil 
\begin{equation*}
    \gls{delta}\gls{X_bar} = \begin{Bmatrix}
        \gls{delta} \overline{X} & \gls{delta}\overline{Y} & \gls{delta}\overline{Z}
    \end{Bmatrix}^\mathrm{T}
\end{equation*}
a virtuálních povrchových sil 
\begin{equation*}
    \gls{delta}\gls{p_bar} = \begin{Bmatrix}
        \gls{delta} p_{\gls{x}} & \gls{delta} p_{\gls{y}} & \gls{delta} p_{\gls{z}}
    \end{Bmatrix}^\mathrm{T},
\end{equation*} 
změníme stav zatížení tělesa a připojením virtuálních napětí $$\gls{delta}\gls{sigma} = \begin{Bmatrix}
    \gls{delta}\gls{sigma_i}[\gls{x}] & \gls{delta}\gls{sigma_i}[\gls{y}] & \gls{delta}\gls{sigma_i}[\gls{z}] & \gls{delta}\gls{tau_i}[\gls{y}\gls{z}] & \gls{delta}\gls{tau_i}[\gls{z}\gls{x}] & \gls{delta}\gls{tau_i}[\gls{x}\gls{y}]
\end{Bmatrix}^\mathrm{T},$$ jeho napjatost.

Nemá-li být narušena rovnováha tělesa, musí v~každém bodě platit rovnice \ref{eq:equilibrium_equations}

\begin{equation}
    \boldsymbol\partial \gls{delta}\gls{sigma} + \gls{delta}\gls{X_bar} = \matr{0},
\end{equation}

a na hranici tělesa \gls{gamma}

\begin{equation}
    \gls{normal_matrix} \gls{delta}\gls{sigma} - \gls{delta}\gls{p_bar} = \matr{0}.
\end{equation}

Výsledné síly $(\gls{X_bar} + \gls{delta}\gls{X_bar})$, $(\gls{p_bar} + \gls{delta}\gls{p_bar})$ a výsledná napětí $(\gls{sigma} + \gls{delta}\gls{sigma})$ se nazývají \textbf{staticky přípustné}.

Princip virtuálních prací lze matematicky zapsat následovně
\begin{equation}
    \label{eq:principle_of_virtual_work}
    \small
    \underbrace{
    \int_{\gls{omega}} 
    (\gls{sigma} + \gls{delta}\gls{sigma})^{\mathrm{T}}
    ( \gls{eps} + \gls{delta}\gls{eps} ) 
    \dd{\gls{omega}}}_{\text{virtuální práce vnitřních sil}}
    =
    \underbrace{
    \int_{\gls{omega}}
    (\gls{X_bar} + \gls{delta}\gls{X_bar})^{\mathrm{T}}
    ( \gls{u} + \gls{delta}\gls{u} )
    \dd{\gls{omega}}
    +
    \int_{\gls{gamma}}
    (\gls{p_bar} + \gls{delta}\gls{p_bar})^{\mathrm{T}}
    ( \gls{u} + \gls{delta}\gls{u} )
    \dd{\gls{gamma}}.}_{\text{virtuální práce vnějších sil}}
\end{equation}

Vhodným upravením rovnice \ref{eq:principle_of_virtual_work}, při uvážení vzájemné nezávislosti zavedených virtuálních polí, obdržíme čtyři rovnice, které musí být splněny nezávisle na sobě,

\begin{itemize}
    \item Clapeyronův (divergenční) teorém
    \begin{equation}
        \label{eq:clapeyron}
        \int_{\gls{omega}}\gls{sigma}^{\mathrm{T}} \gls{eps} \dd{\gls{omega}}
        =
        \int_{\gls{omega}}
        \gls{X_bar}^{\mathrm{T}} \gls{u} \dd{\gls{omega}}
        +
        \int_{\gls{gamma}} \gls{p_bar}^{\mathrm{T}} \gls{u} \dd{\gls{gamma}},
    \end{equation}
    \item princip virtuálních posunutí
    \begin{equation}
        \label{eq:principle_of_virtual_displacement}
        \int_{\gls{omega}}\gls{sigma}^{\mathrm{T}} \gls{delta}\gls{eps} \dd{\gls{omega}}
        =
        \int_{\gls{omega}}
        \gls{X_bar}^{\mathrm{T}} \gls{delta}\gls{u} \dd{\gls{omega}}
        +
        \int_{\gls{gamma}} \gls{p_bar}^{\mathrm{T}} \gls{delta}\gls{u} \dd{\gls{gamma}},
    \end{equation}
    \item princip virtuálních sil
    \begin{equation}
        \label{eq:principle_of_virtual_forces}
        \int_{\gls{omega}} \gls{delta}\gls{sigma}^{\mathrm{T}} \gls{eps} \dd{\gls{omega}}
        =
        \int_{\gls{omega}}
        \gls{delta}\gls{X_bar}^{\mathrm{T}} \gls{u} \dd{\gls{omega}}
        +
        \int_{\gls{gamma}} \gls{delta}\gls{p_bar}^{\mathrm{T}} \gls{u} \dd{\gls{gamma}},
    \end{equation}
    \item čtvrtá rovnice, která nemá bezprostřední využití, ve tvaru
    \begin{equation}
        \int_{\gls{omega}} \gls{delta}\gls{sigma}^{\mathrm{T}} \gls{delta}\gls{eps} \dd{\gls{omega}}
        =
        \int_{\gls{omega}}
        \gls{delta}\gls{X_bar}^{\mathrm{T}} \gls{delta}\gls{u} \dd{\gls{omega}}
        +
        \int_{\gls{gamma}} \gls{delta}\gls{p_bar}^{\mathrm{T}} \gls{delta}\gls{u} \dd{\gls{gamma}}.
    \end{equation}
\end{itemize}


\subsection{Princip virtuálních posunutí}

Dále se budeme zabývat principem virtuálních posunutí, ze kterého odvodíme algoritmus deformační metody.

Uvažujme takové virtuální posuny \gls{delta}\gls{u}, které nenarušují kinematické okrajové podmínky
\begin{equation}
    \label{eq:virtual_displacements_boundary}
    \gls{delta}\gls{u} = \matr{0}, \quad \text{na hranici \gls{gamma_u}},
\end{equation}

a zároveň splňují geometrické rovnice uvnitř tělesa
\begin{equation}
    \label{eq:pvd_geometric_eqs}
    \gls{delta}\gls{eps} = \boldsymbol{\partial}\gls{delta}\gls{u}, \quad \text{uvnitř tělesa \gls{omega}}.
\end{equation}

Rovnici \ref{eq:principle_of_virtual_displacement} zapíšeme s~přihlédnutím k~\ref{eq:virtual_displacements_boundary},
\begin{equation}
    \label{eq:pvd_start}
    \int_{\gls{omega}}\gls{sigma}^{\mathrm{T}} \gls{delta}\gls{eps} \dd{\gls{omega}}
    =
    \int_{\gls{omega}}
    \gls{X_bar}^{\mathrm{T}} \gls{delta}\gls{u} \dd{\gls{omega}}
    +
    \int_{\gls{gamma_p}} \gls{p_bar}^{\mathrm{T}} \gls{delta}\gls{u} \dd{\gls{gamma_p}}.
\end{equation}

Levou stranu rovnice \ref{eq:pvd_start} upravíme dosazením z~rovnice \ref{eq:pvd_geometric_eqs},

\begin{equation}
    \int_{\gls{omega}}\gls{sigma}^{\mathrm{T}} \gls{delta}\gls{eps} \dd{\gls{omega}}
    =
    \int_{\gls{omega}}\gls{sigma}^{\mathrm{T}} \boldsymbol\partial \gls{delta}\gls{u} \dd{\gls{omega}},
\end{equation}

a zintegrujeme per partes, pomocí věty o~integraci tenzorového pole \cite[55]{prpe20},

\begin{equation}
    \label{eq:pvd_per_partes}
    \int_{\gls{omega}}\gls{sigma}^{\mathrm{T}} \boldsymbol\partial \gls{delta}\gls{u} \dd{\gls{omega}}
    =
    \int_{\gls{gamma}} \gls{delta}\gls{u}^{\mathrm{T}} \gls{normal_matrix} \gls{sigma} \dd{\gls{gamma}}
    -
    \int_{\gls{omega}} \gls{delta}\gls{u}^{\mathrm{T}} \boldsymbol\partial \gls{sigma} \dd{\gls{omega}}.
\end{equation}

Dosazením z~rovnice \ref{eq:pvd_per_partes} do původního vztahu \ref{eq:pvd_start} dostaneme rovnici

\begin{equation}
    \int_{\gls{gamma}} \gls{delta}\gls{u}^{\mathrm{T}} \gls{normal_matrix} \gls{sigma} \dd{\gls{gamma}}
    -
    \int_{\gls{omega}} \gls{delta}\gls{u}^{\mathrm{T}} \boldsymbol\partial \gls{sigma} \dd{\gls{omega}}
    =
    \int_{\gls{omega}}
    \gls{X_bar}^{\mathrm{T}} \gls{delta}\gls{u} \dd{\gls{omega}}
    +
    \int_{\gls{gamma_p}} \gls{p_bar}^{\mathrm{T}} \gls{delta}\gls{u} \dd{\gls{gamma_p}},
\end{equation}

rozepsáním prvního integrálu na část s~předepsaným zatížením \gls{gamma_p} a s~předepsanými posuny \gls{gamma_u} dostaneme tvar

\begin{equation}
    \small
    \int_{\gls{gamma_p}} \gls{delta}\gls{u}^{\mathrm{T}} \gls{normal_matrix} \gls{sigma} \dd{\gls{gamma_p}}
    +
    \int_{\gls{gamma_u}} \gls{delta}\gls{u}^{\mathrm{T}} \gls{normal_matrix} \gls{sigma} \dd{\gls{gamma_u}}
    -
    \int_{\gls{omega}} \gls{delta}\gls{u}^{\mathrm{T}} \boldsymbol\partial \gls{sigma} \dd{\gls{omega}}
    =
    \int_{\gls{omega}}
    \gls{X_bar}^{\mathrm{T}} \gls{delta}\gls{u} \dd{\gls{omega}}
    +
    \int_{\gls{gamma_p}} \gls{p_bar}^{\mathrm{T}} \gls{delta}\gls{u} \dd{\gls{gamma_p}},
\end{equation}

kde podle kinematické okrajové podmínky \ref{eq:virtual_displacements_boundary}

\begin{equation*}
    \int_{\gls{gamma_u}} \gls{delta}\gls{u}^{\mathrm{T}} \gls{normal_matrix} \gls{sigma} \dd{\gls{gamma_u}} = 0.
\end{equation*}

Po úpravě dostaneme

\begin{equation}
    \label{eq:pvd}
    \int_{\gls{omega}} \gls{delta}\gls{u}^{\mathrm{T}} 
    (\boldsymbol{\partial} \gls{sigma} + \gls{X_bar})
    \dd{\gls{omega}}
    +
    \int_{\gls{gamma_p}} \gls{delta}\gls{u}^{\mathrm{T}}
    (-\gls{normal_matrix} \gls{sigma}
    +
    \gls{p_bar})
    \dd{\gls{gamma_p}}
    = 0.
\end{equation}

Protože virtuální posuny \gls{delta}\gls{u} jsou libovolné, rovnice \ref{eq:pvd} bude splněna právě tehdy, pokud budou zároveň platit statické rovnice \ref{eq:equilibrium_equations} a statické okrajové podmínky \ref{eq:static_boundary_conditions}. 